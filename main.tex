\documentclass[12pt]{article}
\usepackage[utf8]{inputenc}
\usepackage[T1]{fontenc}
\usepackage{mlmodern}
\usepackage[a4paper, margin=3cm]{geometry}
\usepackage{enumerate}

\usepackage[pdftitle=Riemann-Roch\ Through\ Cohomology,
pdfauthor=Miika\ Rankaviita,pdfsubject=Sheaf\ Cohomology,
colorlinks,linkcolor=black,citecolor=blue]{hyperref}

\usepackage{mathtools}
\usepackage{amsmath}
\usepackage{amssymb}
\usepackage{amsthm}
\usepackage{calrsfs}
\usepackage{cancel}
\usepackage{float}
\usepackage{tikz}
\usepackage{tikz-cd}

\usepackage{tcolorbox}
\usepackage{emoji}

\linespread{1.1}
\frenchspacing

\begin{document}

\title{Riemann--Roch Through Cohomology\\
        \large A Motivated Introduction to Sheaves and Cohomology}
\author{Miika Rankaviita\\Imperial College London}
\date{\today}
\maketitle

\theoremstyle{plain}
\newtheorem{thm}{Theorem}[section]
\newtheorem{cor}{Corollary}[thm]
\newtheorem{prop}[thm]{Proposition}
\newtheorem{lemm}[thm]{Lemma}
\newtheorem*{rem}{Remark}
\theoremstyle{definition}
\newtheorem{defin}[thm]{Definition}
\newtheorem{ex}[thm]{Example}

\tcbset{beforeafter skip=8pt, toptitle=2pt, bottomtitle=2pt}
\newtcolorbox{cat}{colback=blue!5!white, colframe=blue!80!white, title=\emoji{warning} Abstract nonsense ahead}

\section{Introduction}
In this paper I will give a proof of the Riemann--Roch theorem
by using one of the most important tools in modern algebraic geometry ---
sheaf cohomology. This text is aimed at the student with
basic knowledge of varieties, and will function as an introduction
to the topic of sheaf cohomolgy, motivated by the problem of classifying
algebraic varieties.

% TODO: Insert a short description of Riemann-Roch

The main ingredient in the proof of the Riemann-Roch theorem is
Serre duality, which is a more general result. The proof will again
be cohomological and is usually rather abstract. However, I will
approach the proof by giving concrete interpretations of
the cohomology groups by following Serre's exposition in \cite{serre}.

The reader is assumed to be familiar with basic algebraic geometry.
An excellent introduction to the topic is \cite{reid}, and a good set
of lecture notes which goes beyond the basics is \cite{gathmann}.
Through out the paper, I will also try to point out category theoretical
contexts of the concepts discussed. However, it is not necessary to know
any category theory to understand this paper, and I will always place a
warning for people who are not fond of abstract nonsense:

\begin{cat}
  The product $(A\prod B, \pi_{A}, \pi_{B})$ of $A$ and $B$ is defined by the
  following universal property.
  \[
    \begin{tikzcd}
      &C\arrow{ld}[swap]{f}\arrow[dashed]{d}{!\exists}\arrow{rd}{g}&\\
      A&A\prod B\arrow{l}{\pi_{A}}\arrow{r}[swap]{\pi_{B}}&B
    \end{tikzcd}
  \]
\end{cat}

\section{Sheaves}
The category of sheaves is abelian, so we can do homological
algebra in it.

\begin{thm}\label{thm:ses_equivalence}
  For a collection $(\mathcal{F}_{i})_{i\in\mathbb{Z}}$ of sheaves on
  a variety $X$ and maps ${\mathcal{F}_{i}\to\mathcal{F}_{i+1}}$,
  the following are equivalent.
  \begin{enumerate}[(i)]
    \item The sequence
          \[
            \begin{tikzcd}
              \cdots \arrow{r} & \mathcal{F}_{i-1}\arrow{r}
              & \mathcal{F}_{i}\arrow{r} & \mathcal{F}_{i+1}\arrow{r} & \cdots
            \end{tikzcd}
          \]
          is exact.
    \item There is an open cover $(U_{j})_{j\in\mathbb{I}}$ of $X$ such that
          the sequence
          \[
          \begin{tikzcd}
            \cdots \arrow{r} & \mathcal{F}_{i-1}\mid_{U_{j}}\arrow{r}
            & \mathcal{F}_{i}\mid_{U_{j}}\arrow{r}
            & \mathcal{F}_{i+1}\mid_{U_{j}}\arrow{r} & \cdots
          \end{tikzcd}
          \]
          is exact for every $j\in I$.
    \item For every $P\in X$, the induced sequence
          \[
          \begin{tikzcd}
            \cdots \arrow{r} & (\mathcal{F}_{i-1})_{P}\arrow{r}
            & (\mathcal{F}_{i})_{P}\arrow{r}
            & (\mathcal{F}_{i+1})_{P}\arrow{r} & \cdots
          \end{tikzcd}
          \]
          is exact.
  \end{enumerate}
\end{thm}

\section{Sheaf cohomology}
Studying sheaves using homological algebra turns out to be surprisingly
useful in many situations. For example, knowing that there is a SES
\[
  \begin{tikzcd}
    0 \arrow{r} & \mathcal{F} \arrow{r} & \mathcal{G} \arrow{r} &
    \mathcal{H} \arrow{r} & 0
  \end{tikzcd}
\]
lets us relate the three sheaves together. By Thm.~\ref{thm:ses_equivalence}
this information is inherently \emph{local} since this sequence is exact
if and only if the corresponding sequences on stalks are exact.
Then the question is: Can we get \emph{global} information from
such exact sequences? We would hope that just as the sequence is exact
on stalks, it would also be exact on global sections:
\[
  \begin{tikzcd}
    0 \arrow{r} & \Gamma(\mathcal{F}) \arrow{r} & \Gamma(\mathcal{G})
    \arrow{r} & \Gamma(\mathcal{H}) \arrow{r} & 0
  \end{tikzcd}
\]
Unfortunately, this is not the case. One can prove that this sequence
is exact at $\Gamma(\mathcal{F})$ and $\Gamma(\mathcal{G})$, but it
is not always exact at $\Gamma(\mathcal{H})$.
\begin{cat}
  We say that the global sections functor $\Gamma: \text{Sh}(X)\to \mathcal{O}_{X}\text{-Mod}$ is left-exact but not right-exact.
\end{cat}
In more concrete terms, if we know that a morphism $f: \mathcal{F}
\to\mathcal{G}$ is injective on stalks, it is also injective on
global sections. But if the morphism is surjective on stalks, we don't
know whether or not it is surjective on global sections.

Although exact sequences are not completely preserved under taking
global sections, we won't give up! There might still be \emph{a way of
measuring how much exactness fails}. We could measure the
\emph{obstruction} to exactness by continuing the sequence to the right
so that the following sequence is exact.
\[
  \begin{tikzcd}
    0 \arrow{r} & \Gamma(\mathcal{F}) \arrow{r} & \Gamma(\mathcal{G})
    \arrow{r} & \Gamma(\mathcal{H})
    \arrow[out=0, in=180, looseness=1.5, overlay]{dll} & \\
     & H^{1}(\mathcal{F}) \arrow{r} & H^{1}(\mathcal{G})
    \arrow{r} & H^{1}(\mathcal{H}) \arrow{r} & \cdots \\
    \cdots \arrow{r}& H^{k}(\mathcal{F}) \arrow{r} & H^{k}(\mathcal{G})
    \arrow{r} & H^{k}(\mathcal{H}) \arrow{r} & 0
  \end{tikzcd}
\]

\section{The Riemann-Roch theorem}
\subsection{Applications}

\section{Serre duality}

\section{Bibliography}
\bibliographystyle{alpha}
\bibliography{refs}

\end{document}
