\documentclass[12pt]{article}
\usepackage[utf8]{inputenc}
\usepackage[T1]{fontenc}
\usepackage{mlmodern}
\usepackage[a4paper, margin=3cm]{geometry}
\usepackage[shortlabels]{enumitem}

\usepackage[pdftitle=Riemann-Roch\ Through\ Cohomology,
pdfauthor=Miika\ Rankaviita,pdfsubject=Sheaf\ Cohomology,
colorlinks,linkcolor=black,citecolor=blue]{hyperref}

\usepackage{mathtools}
\usepackage{amsmath}
\usepackage{amssymb}
\usepackage{amsthm}
\usepackage{mathrsfs}
\usepackage{cancel}
\usepackage{braket}

\DeclareMathOperator{\im}{im}
\DeclareMathOperator{\ord}{ord}
\DeclareMathOperator{\res}{Res}

\usepackage{float}
\usepackage{tikz}
\usepackage{tikz-cd}

\usepackage{tcolorbox}
\usepackage{emoji}

\linespread{1.1}
\frenchspacing

\begin{document}

\title{Riemann-Roch Through Cohomology\\
        \large A Motivated Introduction to Sheaves and Cohomology}
\author{Miika Rankaviita\\Imperial College London}
\date{\today}
\maketitle

\theoremstyle{plain}
\newtheorem{thm}{Theorem}[section]
\newtheorem{cor}{Corollary}[thm]
\newtheorem{prop}[thm]{Proposition}
\newtheorem{lemm}[thm]{Lemma}
\newtheorem*{rem}{Remark}
\theoremstyle{definition}
\newtheorem{defin}[thm]{Definition}
\newtheorem{ex}[thm]{Example}

\tcbset{beforeafter skip=8pt, toptitle=2pt, bottomtitle=2pt}
\newtcolorbox{cat}{colback=blue!5!white, colframe=blue!80!white, title=\emoji{warning} Abstract nonsense ahead}
\newtcolorbox{lnote}{colback=white, colframe=yellow!90!black, center}
\newtcbox{\note}{colback=white, colframe=yellow!90!black, center}

\newcommand{\diffs}{H^0(X,K_X\otimes\mathcal{L}(D)^{\vee})}

\section{Introduction}
In this paper I will give a proof of the Riemann-Roch theorem
by using one of the most important tools in modern algebraic geometry ---
sheaf cohomology. This text is aimed at the student with
basic knowledge of varieties, and will function as an introduction
to the topic of sheaf cohomolgy, motivated by the problem of classifying
algebraic curves.

% TODO: Insert a short description of Riemann-Roch

The main ingredient in the proof of the Riemann-Roch theorem is
Serre duality, which is a more general result. The proof will again
be cohomological and is usually rather abstract. However, I will
approach the proof by giving concrete interpretations of
the cohomology groups by following Serre's exposition in \cite{serre}.

The reader is assumed to be familiar with basic algebraic geometry.
An excellent introduction to the topic is \cite{reid}, and a good set
of lecture notes which goes beyond the basics is \cite{gathmann}.
Through out the paper, I will also try to point out category theoretical
contexts of the concepts discussed. However, it is not necessary to know
any category theory to understand this paper, and I will always place a
warning for people who are not fond of abstract nonsense.

\section{Sheaves}
The category of sheaves which takes values in an abelian category is
itself abelian, so we can do homological algebra in it.

\begin{thm}\label{thm:ses_equivalence}
  For a collection $(\mathscr{F}_{i})_{i\in\mathbb{Z}}$ of sheaves on
  a variety $X$ and maps ${\mathscr{F}_{i}\to\mathscr{F}_{i+1}}$,
  the following are equivalent.
  \begin{enumerate}[(i)]
    \item The sequence
          \[
          \begin{tikzcd}
            \cdots \arrow{r} & \mathscr{F}_{i-1}\arrow{r}
            & \mathscr{F}_{i}\arrow{r} & \mathscr{F}_{i+1}\arrow{r} & \cdots
          \end{tikzcd}
          \]
          is exact.
    \item There is an open cover $(U_{j})_{j\in\mathbb{I}}$ of $X$ such that
          the sequence
          \[
          \begin{tikzcd}
            \cdots \arrow{r} & \mathscr{F}_{i-1}\mid_{U_{j}}\arrow{r}
            & \mathscr{F}_{i}\mid_{U_{j}}\arrow{r}
            & \mathscr{F}_{i+1}\mid_{U_{j}}\arrow{r} & \cdots
          \end{tikzcd}
          \]
          is exact for every $j\in I$.
    \item For every $P\in X$, the induced sequence
          \[
          \begin{tikzcd}
            \cdots \arrow{r} & (\mathscr{F}_{i-1})_{P}\arrow{r}
            & (\mathscr{F}_{i})_{P}\arrow{r}
            & (\mathscr{F}_{i+1})_{P}\arrow{r} & \cdots
          \end{tikzcd}
          \]
          is exact.
  \end{enumerate}
\end{thm}

\section{Sheaf cohomology}
Studying sheaves using homological algebra turns out to be surprisingly
useful in many situations. For example, knowing that there is a SES
\[
  \begin{tikzcd}
    0 \arrow{r} & \mathscr{F} \arrow{r} & \mathscr{G} \arrow{r} &
    \mathscr{H} \arrow{r} & 0
  \end{tikzcd}
\]
lets us relate the three sheaves together. By Thm.~\ref{thm:ses_equivalence}
this information is inherently \emph{local} since this sequence is exact
if and only if the corresponding sequences on stalks are exact.
Then the question is: Can we get \emph{global} information from
such exact sequences? We would hope that just as the sequence is exact
on stalks, it would also be exact on global sections:
\[
\begin{tikzcd}
  0 \arrow{r} & \Gamma(\mathscr{F}) \arrow{r} & \Gamma(\mathscr{G})
  \arrow{r} & \Gamma(\mathscr{H}) \arrow{r} & 0.
\end{tikzcd}
\]
Unfortunately, this is not the case. One can prove that this sequence
is exact at $\Gamma(\mathscr{F})$ and $\Gamma(\mathscr{G})$, but it
is not always exact at $\Gamma(\mathscr{H})$.
\begin{cat}
  We say that the global sections functor $\Gamma: \text{Sh}(X)\to \textbf{Ab}$ is left-exact but not right-exact.
\end{cat}
In more concrete terms, if we know that a morphism $f: \mathscr{F}
\to\mathscr{G}$ is injective on stalks, it is also injective on
global sections. But if the morphism is surjective on stalks, we don't
know whether or not it is surjective on global sections.
% TODO: Add an example

Although exact sequences are not completely preserved under taking
global sections, we won't give up! There might still be \emph{a way of
measuring how much exactness fails}. We could measure the
\emph{obstruction} to exactness by continuing the sequence to the right
so that the following sequence is exact.
\[
\begin{tikzcd}
  0 \arrow{r} & \Gamma(\mathscr{F}) \arrow{r} & \Gamma(\mathscr{G})
  \arrow{r} & \Gamma(\mathscr{H})
  \arrow[out=0, in=180, looseness=1.5, overlay]{dll} & \\
    & H^{1}(\mathscr{F}) \arrow{r} & H^{1}(\mathscr{G})
  \arrow{r} & H^{1}(\mathscr{H}) \arrow{r} & \cdots \\
  \cdots \arrow{r}& H^{i}(\mathscr{F}) \arrow{r} & H^{i}(\mathscr{G})
  \arrow{r} & H^{i}(\mathscr{H}) \arrow{r} & \cdots.
\end{tikzcd}
\]
This problem of extending incomplete short exact sequences appears
elsewhere in homological algebra and is generally solved by constructing
so-called \emph{derived functors}. These modules $H^{i}(-)$ given by derived
functors are then called the sheaf cohomology groups. In practise, they are
difficult to compute, so we want to find an alternative definition,
which is easier to work with. This is achieved by \emph{\v Cech cohomology}.
Next, I will explain derived functors and show how we arrived at \v Cech
cohomology from there. However, this will involve a lot of new, high-level
concepts and category theory. Understanding the philosophy behind
\v Cech cohomolgy is not essential, and one can safely skip straight to the
definition of the \v Cech cohomology groups given in Def.~\ref{def:cech}.

\subsection{Motivating \v Cech cohomology}


\subsection{Results in cohomology}

\begin{prop}\label{prop:const_sheaf}
  If $X$ is an irreducible variety, and $A$ is an abelian group,
  then for the constant sheaf $\underline{A}$,
  \begin{enumerate}[(a)]
    \item $H^{0}(X,\underline{A}) = A$,
    \item $H^{1}(X,\underline{A})=0$.
  \end{enumerate}
\end{prop}


\section{The Riemann-Roch theorem}
Equiped with sheaf cohomology, we will prove the Riemann-Roch theorem
using the methods we have learnt. But first we need to quickly review
two constructions needed to state the Riemann-Roch theorem: \emph{divisors}
and \emph{differentials}.

\subsection{Divisors and differentials}

\subsection{Proof of Riemann-Roch}
We are now able to state and prove an incomplete version of the Riemann-Roch
theorem, which we will make complete after proving Serre duality.

\begin{lnote}
  In this section, $X$ will always denote an irreducible, non-singular,
  complete algebraic curve, and any divisor $D$ is defined on $X$.
\end{lnote}

% TODO: Give a proof that the cohomology groups are finite-dimensional.

\begin{thm}[Riemann-Roch, incomplete version]
  \label{thm:riemann_roch_incomplete}
  For every divisor $D$,
  \[
    h^{0}(X, \mathcal{L}(D))-h^{1}(X, \mathcal{L}(D))=\deg(D)+1-g,
  \]
  where $g=h^{1}(X, \mathcal{O}_{X})$.
\end{thm}
\begin{proof}
  We can use an induction argument, because any divisor $D$ can be
  obtained from the zero divisor by adding and subtracting points.
  Thus, we proceed by first proving the base case and then proving
  the induction step.

  \begin{description}[style=nextline]
    \item[base case$\big)$]
          Since $\mathcal{L}(0)=\mathscr{O}_{X}$ and $\deg(0)=0$,
          we need to verify that
          \[h^{0}(X, \mathscr{O}_{X})-h^{1}(X, \mathscr{O}_{X})=1-g.\]
          But note that the only globally defined regular functions
          on $X$ are constant by Cor.~\ref{cor:global_const},
          and thus they form a one-dimensional vector space.
          Moreover, $h^{1}(X, \mathscr{O}_{X})=g$ by definition
          so that the equality holds.
    \item[induction step$\big)$]
          In the induction step we want to relate the zeroth and first
          cohomology groups of $\mathcal{L}(D)$ to the zeroth and
          first cohomology groups of $\mathcal{L}(D+P)$, where
          $P$ is some point. To do this, we first note that
          $\mathcal{L}(D)$ is a subsheaf of $\mathcal{L}(D+P)$, since
          the orders of germs of $\mathcal{L}(D+P)$ at $P$ are allowed to
          be smaller than the orders of germs of $\mathcal{L}(D)$ at $P$.
          Thus, there is an exact sequence
          \[
          \begin{tikzcd}
            0\arrow{r} & \mathcal{L}(D)\arrow{r} & \mathcal{L}(D+P)\arrow{r}
            & Q\arrow{r} & 0,
          \end{tikzcd}
          \]
          where $Q$ is the quotient sheaf. The stalks of $Q$ are clearly
          zero away from $P$. The stalk at $P$ consists of zero and
          elements of the form $u/t^{n+1}$, where $t$ is the local
          uniformiser of $\mathscr{O}_{P}$, $u$ is a unit in
          $\mathcal{O}_{P}$, and $n$ is the order of $P$ in $D$.
          As $\mathcal{O}_{P}/(t)=k$, we can write $u=vt+r$,
          where $v\in\mathcal{O}_{P}$ and $r\in k$.
          Then,
          \[\frac{u}{t^{n+1}}=\frac{v}{t^n}+\frac{r}{t^{n+1}}.\]
          Since $v/t^n$ is an element of $\mathcal{L}(D)_{P}$, we conclude
          that every element of $\mathcal{L}(D+P)_{P}$ is equivalent to
          an element $r/t^{n+1}$ modulo $\mathcal{L}(D)_{P}$ for some
          $r\in k$. Therefore, $Q_{P}\cong k$ and $Q$ is the skyscraper sheaf!

          Now we apply our cohomology machinery on the SES
          \[
          \begin{tikzcd}
            0\arrow{r} & \mathcal{L}(D)\arrow{r} & \mathcal{L}(D+P)\arrow{r}
            & k_{P}\arrow{r} & 0
          \end{tikzcd}
          \]
          to get the following exact sequence (using
          Prop.~\ref{prop:sky_cohom}).
          \[
          \begin{tikzcd}
            0\arrow{r} & H^{0}(X, \mathcal{L}(D))\arrow{r}
            & H^{0}(X, \mathcal{L}(D+P))\arrow{r}
            & H^{0}(X, k_{P}) \\
            \arrow{r} & H^{1}(X, \mathcal{L}(D))\arrow{r}
            & H^{1}(X, \mathcal{L}(D+P))\arrow{r} & 0.
          \end{tikzcd}
          \]
          Now, by Prop.~\ref{prop:homology_dim},
          \[
          h^{0}(X,\mathcal{L}(D))-h^{0}(X, \mathcal{L}(D+P))
          +1-h^{1}(X,\mathcal{L}(D))+h^{1}(X,\mathcal{L}(D+P)) = 0.
          \]
          Therefore,
          \begin{align*}
            h^{0}(X,\mathcal{L}(D+P))&-h^{1}(X,\mathcal{L}(D+P))
            =\left(h^{0}(X,\mathcal{L}(D))-h^{1}(X,\mathcal{L}(D+P))\right)
              +1 \\
            =&\deg(D)+1-g+1\quad\text{(by induction hypothesis)} \\
            =&\deg(D+P)+1-g.
          \end{align*}
          This is exactly the induction step we wanted to prove.
          We also need to prove
          \[
            h^{0}(X,\mathcal{L}(D-P))-h^{1}(X,\mathcal{L}(D-P))
            =\deg(D-P)+1-g,
          \]
          but we can run the same argument starting with the SES
          \[
          \begin{tikzcd}
            0\arrow{r} & \mathcal{L}(D-P)\arrow{r} & \mathcal{L}(D)\arrow{r}
            & k_{P}\arrow{r} & 0.
          \end{tikzcd}
          \]
  \end{description}
\end{proof}

This form of the theorem is not the most useful one, because computing
$h^{1}(X,\mathcal{L}(D))$ is not easy. Luckily, the Serre Duality theorem
--- which we will prove later --- relates zeroth and first cohomology
groups, and implies the following equality of dimensions:
\[h^{1}(X,\mathcal{L}(D))=h^{0}(X,\mathcal{L}(K_{X}-D))\]

Now, this equality lets us write the complete form of the Riemann-Roch
theorem.
\begin{thm}[Riemann-Roch, complete version]\label{thm:riemann_roch_complete}
  For every divisor $D$,
  \[
    h^{0}(X, \mathcal{L}(D))-h^{0}(X, \mathcal{L}(K_{X}-D))=\deg(D)+1-g,
  \]
  where $g=h^{1}(X, \mathcal{O}_{X})$.
\end{thm}

\subsection{Applications}
Before proving the Serre duality theorem, I want to take some time to
look at applications of the Riemann-Roch theorem.

\section{Serre duality}
The rest of this paper is devoted to proving the Serre duality:
\begin{thm}[Serre Duality]
  If $X$ is an algebraic curve as in the previous section
  and $D$ is a divisor on $X$, there is an isomorphism
  \[
    H^{1}(X, \mathcal{L}(D))^{\vee}\cong H^{0}(X, K_{X}
    \otimes \mathcal{L}(D)^{\vee}).
  \]
  of $k$-vector spaces, where $K_{X}$ is the canonical divisor of $X$.
\end{thm}
We will prove the theorem by first finding a more concrete representation of
$H^{1}(X,\mathcal{L}(D))$ and then constructing a perfect pairing between
$H^{1}(X,\mathcal{L}(D))$ and $H^{0}(X,K_{X}\otimes\mathcal{L}(D)^{\vee})$,
which will give us the isomorphism.

\subsection{Concrete representation of $H^{1}(X,\mathcal{L}(D))$}
To prove the Serre duality, we do not want to directly work with the \v Cech
cohomology definition of $H^{1}(X,\mathcal{L}(D))$.
Instead we want to find some SES involving $\mathcal{L}(D)$,
take the cohomology sequence of the SES, and then use it to simplify
the definition of $H^{1}(X,\mathcal{L}(D))$. Since $\mathcal{L}(D)$ is
a subsheaf of the constant sheaf $\underline{k(X)}$, we can simply consider
the following SES.
\[
  \begin{tikzcd}
    0\arrow{r} & \mathcal{L}(D)\arrow{r} & \underline{k(X)}\arrow{r}
    & \underline{k(X)}/\mathcal{L}(D)\arrow{r} & 0,
  \end{tikzcd}
\]
which yields the following exact sequence
\[
  \begin{tikzcd}
    H^{0}(X,\underline{k(X)})\arrow{r} & H^{0}(X,\underline{k(X)}
    /\mathcal{L}(D))\arrow{r} & H^{1}(X,\mathcal{L}(D))\arrow{r}
    & H^{1}(X,\underline{k(X)}).
  \end{tikzcd}
\]
But Prop.~\ref{prop:const_sheaf} implies that
$H^{0}(X,\underline{k(X)})=k(X)$ and $H^{1}(X,\underline{k(X)})=0$
so that the exact sequence simplifies to
\[
  \begin{tikzcd}
    k(X)\arrow{r} & H^{0}(X,\underline{k(X)}/\mathcal{L}(D))\arrow{r}
    & H^{1}(X,\mathcal{L}(D))\arrow{r} & 0.
  \end{tikzcd}
\]
Let us first try to understand the space
$H^{0}(X,\underline{k(X)}/\mathcal{L}(D))$. An element of the space
is of the form $([f_{P}])_{P\in X}$, where $[f_{P}]$ is the equivalence
class of some $f_{P}\in k(X)$ modulo $\mathcal{L}(D)_{P}$.
Note that the $f_{P}$ must be related together so that the sections
satisfy sheaf axioms, and it would be easier if we didn't need to worry about
this condition. We can in fact construct an isomorphic vector space, which is
similar, but where we don't need to worry about the sheaf axioms. To see
this, first make the following observation.
\begin{prop}
  If $(f_{P})_{P\in X}\in \Gamma(\underline{k(X)})$,
  $f_{P}\in\mathscr{O}_{X,P}$ for almost all $P\in X$.
\end{prop}
\begin{proof}
  Fix an arbitrary $P\in X$. Since $\underline{k(X)}$ is obtained by
  sheafification of the constant presheaf, there must be an open
  set $U\subseteq X$ and $g\in k(X)$ such that
  $\forall Q\in U,\ [f_{Q}]=[g]$. Next we note that $g$
  is defined for almost all points of $U$ so that $g_{Q}
  \in\mathscr{O}_{X,Q}$ for almost all $Q\in V$.

  Thus, there is an open neighbourhood for every point
  of $X$ where the statement holds. Next we extend this to the whole of $X$.
  Let us construct a sequence of open sets inductively:
  \begin{enumerate}
    \item Choose an arbitrary point $P_{0}\in X$
    \item Let $U_{0}$ be the neighbourhood of $P_{0}$ constructed as above
    \item Assume we have constructed the set $U_{n}$
    \item Choose an arbitrary point $P_{n+1}$ of $X\setminus U_{n}$
    \item Let $U_{n+1}$ be the union of $U_{n}$ with the
          neighbourhood of $P_{n+1}$ constructed as above
  \end{enumerate}
  % Insert picture with X, P_0, U_0, and P_1
  Since $X$ is Noetherian, there must be $N\in\mathbb{N}$
  such that $\forall k\geq N, U_{k+1}=U_{k}$. Moreover, these sets must
  cover $X$, because otherwise the chain wouldn't end at $U_{N}$.
  Therefore, $X$ can be covered by finitely many sets where the statement
  holds.
\end{proof}
Now, we define a vector space of families $\{r_{P}\}_{P\in X}$, where we don't
impose any other requirement on $r_{P}$ other than that $r_{P}\in k(X)$ and
$r_{P}\in\mathscr{O}_{X,P}$ for almost all $P\in X$ (Serre calls such a
family a \emph{r\'epartition}).
% TODO: It might be wise to point out that by r_P\in O_{X,P} we
% actually mean (r_P)_P\in O_{X,P}.
Now, I claim that there is a subspace
$S\leq R$ such that $H^{0}(X,\underline{k(X)}/\mathcal{L}(D))\cong R/S$.
This isomorphism will be given by the trivial map
\[
  \varphi: H^{0}(X,\underline{k(X)}/\mathcal{L}(D))\to R/S
  :([f_{P}])_{P\in X}\mapsto [\{f_{P}\}_{P\in X}].
\]
For this map to be well-defined, $([f_{P}])_{P\in X}$ needs to be
mapped to $[0]$ whenever $f_{P}\in\mathcal{L}(D)_{P}$. Since I want
$\varphi$ also to be injective, such elements $(f_{P})_{P\in X}$ should be
the \emph{only} elements that get mapped to $[0]$. Thus, I define $S$
to be the subspace such that $\set{\{r_{P}\}_{P\in X}\mid \ord_{P}(r)
  \geq -D(P)}=:R(D)$, and I claim that this is the right choice of $S$.

\begin{lemm}
  For a divisor $D$ on a curve $X$, we have the following isomorphism:
  \[
    H^{0}(X,\underline{k(X)}/\mathcal{L}(D))\cong R/R(D).
  \]
\end{lemm}
\begin{proof}
  The map $\varphi$ is a well-defined injection by construction,
  so we only need to show it is surjective. Thus,
  suppose $[\{r_{P}\}_{P\in X}]\in R/R(D)$. I want to show that equivalence
  classes $[r_{P}]$ of the components form a global section of the sheaf
  $\underline{k(X)}/\mathcal{L}(D)$. First, let $P_{1},\ldots,P_{r}$
  be the points where $D$ is non-zero and $Q_{1},\ldots,Q_{s}$ be the
  points where $r_{Q_{i}}\not\in \mathscr{O}_{X,Q_{i}}$. Then, let $P\in X$
  be an arbitrary point. We want to find an open neighbourhood $U\ni P$
  and a section $g\in k(X)$ such that $\forall Q\in U,\ [r_{Q}]=[g]$.
  There are two cases:
  \begin{description}[style=nextline]
    \item[$P\not\in\set{P_{1},\ldots,P_{r},Q_{1},\ldots,Q_{s}}\big)$]
          It follows from the definition of the points $P_{i}$ and $Q_{j}$
          that $[r_{P}]=[0]$. And if we let $U$ be the complement of
          the set $\set{P_{1},\ldots,P_{r},Q_{1},\ldots,Q_{s}}$, we see
          that the same hold for every $r_{Q}$ on $U$ so that we can simply
          choose $0\in k(X)$ as the section on the open neighbourhood $U$.
    \item[$P\in\set{P_{1},\ldots,P_{r},Q_{1},\ldots,Q_{s}}\big)$]
          First, denote $Y=\set{P_{1},\ldots,P_{r},Q_{1},\ldots,Q_{s}}
          \setminus \set{P}$ and $g=r_{P}\in k(X)$. Next, let
          $S_{1},\ldots,S_{t}$ be the points where
          $g_{S_{i}}\not\in\mathscr{O}_{X,S_{i}}$. Then, let $U$ be
          the complement of $Y\cup \set{S_{1},\ldots,S_{t}}$. As above,
          $[r_{Q}]=[0]$ for all $Q\in U$ except for $Q=P$. But since
          the points $S_{i}$ are also included in the complement, we have
          that $[g]=[0]$ away from $P$. Thus, $[r_{Q}]=[g]$ on $U$.
  \end{description}
\end{proof}
Now we can return to the SES derived above and replace
$H^{0}(X,\underline{k(X)}/\mathcal{L}(D))$ by $R/R(D)$:
\[
  \begin{tikzcd}
    k(X)\arrow{r} & R/R(D)\arrow{r} & H^{1}(X,\mathcal{L}(D))\arrow{r} & 0.
  \end{tikzcd}
\]
This exact sequence finally gives us the representation of the first
cohomology group: The second map of the sequence is a surjection
onto $H^{1}(X,\mathcal{L}(D))$. The space $k(X)$ can be seen as a subspace
of $R$ and it is thus the kernel of the map. Such a surjection maps $R/R(D)$
onto the space $R\,/\left(R(D)+k(X)\right)$, and thus we get the isomorphism
\[H^{1}(X,\mathcal{L}(D))\cong R/\left(R(D)+k(X)\right).\]
Now, the dual space $H^{1}(X,\mathcal{L}(D))^{\vee}$ is simply the space
of linear functionals on $R$, which vanish on $R(D)$ and $k(X)$!

\subsection{Constructing a pairing}
Next I will construct a bilinear form
\[
  \langle -,-\rangle:\diffs\times H^{1}(X,\mathcal{L}(D))\to k.
\]
Note that the space $\diffs$ consists of differential forms $\omega$
such that $(\omega)\geq D$.
%But first we want to understand the space $H^{0}(X,K_{X}
%\otimes\mathcal{L}(D)^{\vee})$ better.
%Consider the stalk of the sheaf
%$K_{X}\otimes\mathcal{L}(D)^{\vee}$ at a point $P$:
%\[
%  \left(K_{X}\otimes\mathcal{L}(D)^{\vee}\right)_{P}
%  =(K_{X})_{P}\otimes_{\mathscr{O}_{X}}\mathcal{L}(D)^{\vee}_{P}
%  =D_{k}(\mathscr{O}_{X})\otimes_{\mathscr{O}_{X}}\mathcal{L}(D)^{\vee}_{P}
%  =D_{k}\left(\mathcal{L}(D)^{\vee}_{P}\right).
%\]
Now, define the bilinear form as follows.
\[
  \langle\omega,r\rangle=\sum_{P\in X}\res_{P}(r_{P}\omega),
\]
where $r=[\{r_{P}\}_{P\in X}]\in R\,/\left(R(D)+k(X)\right)$. The sum is
well-defined, because the term $\res_{P}(r_{P}\omega)$ can be non-zero only
when $P$ is a point such that $D(P)\neq 0$ or $r_{P}\not\in\mathscr{O}_{X,P}$.
Otherwise, $r_{P},f\in\mathscr{O}_{X,P}$, if we write $\omega=f\,dt$
where $t$ is a local uniformiser at $P$. This clearly implies that the
coefficients of the negative terms in the serier expansion of $r_{P}f$
are all zero.

Now, Serre duality will follow from showing that the map
\[
  i_{D}:\diffs\to H^{1}(X,\mathcal{L}(D))^{\vee}
  :\omega\mapsto\langle\omega,-\rangle
\]
is a bijection. But first we need to confirm that $i_{D}$ actually maps
differentials to elements of $H^{1}(X,\mathcal{L}(D))^{\vee}$. For a
differential $\omega\in\diffs$, $i_{D}(\omega)$ is of course a linear
functional on $R$, but we need to check that it vanishes on $R(D)$ and
$k(X)$. If $r\in R(D)$, then $(r_{P}\omega)=(r_{P})+(\omega)\geq -D+D=0$
and thus $\res_{P}(r_{P}\omega)=0$ by the same argument as in the last
paragraph. And if $r\in k(X)$, then $\langle r,\omega\rangle=0$ by
Thm.~\ref{thm:residue}. Thus, $i_{D}(\omega)\in H^{1}(X,\mathcal{L}(D))^{\vee}$
for every $\omega\in\diffs$.

Now, we can proceed to prove the bijectivity of $i_{D}$, starting with
injectivity.
\begin{lemm}
  The map $i_{D}:\diffs\to H^{1}(X,\mathcal{L}(D))^{\vee}$ is an injection.
\end{lemm}
\begin{proof}
  The map is injective if its kernel is trivial. Thus, suppose
  $i_{D}(\omega)=0$. I will show that for every $P\in X$,
  $\ord_{P}(\omega)=\infty$, which implies $\omega=0$.
  Assume to the contrary, so that there is a point $P\in X$ where
  $\ord_{P}(\omega)$ is bounded. But now we can construct a r\'epartition
  $r=[\{r_{Q}\}_{Q\in X}]$ such that $r_{Q}=0$ when $Q\neq P$ and
  $r_{P}=1/t^{\ord_{P}(\omega)+1}$, where $t$ is a local uniformiser at $P$.
  Then,
  \[
    i_{D}(\omega)(r)=\sum_{Q\in X}\res_{Q}(r_{Q}\omega)=\res_{P}(r_{P}\omega)
  \]
  Since $\ord_{P}(r_{P}\omega)=\ord_{P}(r_{P})+\ord_{P}(\omega)
  =-\ord_{P}(\omega)-1+\ord_{P}(\omega)=-1$, we have that $i_{D}(\omega)(r)$ is
  non-zero, which contradicts the assumption that $i_{D}(\omega)=0$.
\end{proof}

\section{Bibliography}
\bibliographystyle{alpha}
\bibliography{refs}

\end{document}
