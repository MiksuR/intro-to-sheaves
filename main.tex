\documentclass[12pt, a4paper]{article}
\usepackage[utf8]{inputenc}
\usepackage[T1]{fontenc}
\usepackage{mathtools}
\usepackage{amsmath}
\usepackage{amssymb}
\usepackage{amsthm}
\usepackage{cancel}
\usepackage{float}
\usepackage{tikz}
\usepackage{tikz-cd}

\frenchspacing

\begin{document}

\title{UROP: Riemann-Roch and Serre duality}
\author{Miika Rankaviita\\Imperial College London}
\date{\today}
\maketitle

\section{Introduction}
The main topics of this paper are the Riemann-Roch theorem
and Serre duality, which are extremely important results in
algebraic geometry. We will prove these result using \emph{sheaf
  cohomologoy}, which is a ubiqutous tool in modern algebraic geometry.
The aim of this paper is to develop the machinery of this abstract theory
and use it to prove the two results while linking back to the classical
approach to build intuition. The reader is assumed to be familiar with
basic algebraic geometry. An excellent introduction to the topic is
[reid], and there is also a good set of lecture notes which goes beyond
the basics [gathmann].

\section{Sheaves}
The category of sheaves is abelian, so we can do homological
algebra in it.

\section{Sheaf cohomology}


\section{The Riemann-Roch theorem}
\subsection{Applications}

\section{Serre duality}

\end{document}
