\documentclass[12pt]{article}
\usepackage[utf8]{inputenc}
\usepackage[T1]{fontenc}
\usepackage{mlmodern}
\usepackage[a4paper, margin=3cm]{geometry}
\usepackage[shortlabels]{enumitem}

\usepackage[pdftitle=Riemann-Roch\ Through\ Cohomology,
pdfauthor=Miika\ Rankaviita,pdfsubject=Sheaf\ Cohomology,
colorlinks,linkcolor=black,citecolor=blue]{hyperref}

\usepackage{mathtools}
\usepackage{amsmath}
\usepackage{amssymb}
\usepackage{amsthm}
\usepackage{mathrsfs}
\usepackage{cancel}
\usepackage{float}
\usepackage{tikz}
\usepackage{tikz-cd}

\usepackage{tcolorbox}
\usepackage{emoji}

\linespread{1.1}
\frenchspacing

\begin{document}

\title{Riemann-Roch Through Cohomology\\
        \large A Motivated Introduction to Sheaves and Cohomology}
\author{Miika Rankaviita\\Imperial College London}
\date{\today}
\maketitle

\theoremstyle{plain}
\newtheorem{thm}{Theorem}[section]
\newtheorem{cor}{Corollary}[thm]
\newtheorem{prop}[thm]{Proposition}
\newtheorem{lemm}[thm]{Lemma}
\newtheorem*{rem}{Remark}
\theoremstyle{definition}
\newtheorem{defin}[thm]{Definition}
\newtheorem{ex}[thm]{Example}

\tcbset{beforeafter skip=8pt, toptitle=2pt, bottomtitle=2pt}
\newtcolorbox{cat}{colback=blue!5!white, colframe=blue!80!white, title=\emoji{warning} Abstract nonsense ahead}
\newtcolorbox{lnote}{colback=white, colframe=yellow!90!black, center}
\newtcbox{\note}{colback=white, colframe=yellow!90!black, center}

\section{Introduction}
In this paper I will give a proof of the Riemann-Roch theorem
by using one of the most important tools in modern algebraic geometry ---
sheaf cohomology. This text is aimed at the student with
basic knowledge of varieties, and will function as an introduction
to the topic of sheaf cohomolgy, motivated by the problem of classifying
algebraic curves.

% TODO: Insert a short description of Riemann-Roch

The main ingredient in the proof of the Riemann-Roch theorem is
Serre duality, which is a more general result. The proof will again
be cohomological and is usually rather abstract. However, I will
approach the proof by giving concrete interpretations of
the cohomology groups by following Serre's exposition in \cite{serre}.

The reader is assumed to be familiar with basic algebraic geometry.
An excellent introduction to the topic is \cite{reid}, and a good set
of lecture notes which goes beyond the basics is \cite{gathmann}.
Through out the paper, I will also try to point out category theoretical
contexts of the concepts discussed. However, it is not necessary to know
any category theory to understand this paper, and I will always place a
warning for people who are not fond of abstract nonsense.

\section{Sheaves}
The category of sheaves which takes values in an abelian category is
itself abelian, so we can do homological algebra in it.

\begin{thm}\label{thm:ses_equivalence}
  For a collection $(\mathscr{F}_{i})_{i\in\mathbb{Z}}$ of sheaves on
  a variety $X$ and maps ${\mathscr{F}_{i}\to\mathscr{F}_{i+1}}$,
  the following are equivalent.
  \begin{enumerate}[(i)]
    \item The sequence
          \[
          \begin{tikzcd}
            \cdots \arrow{r} & \mathscr{F}_{i-1}\arrow{r}
            & \mathscr{F}_{i}\arrow{r} & \mathscr{F}_{i+1}\arrow{r} & \cdots
          \end{tikzcd}
          \]
          is exact.
    \item There is an open cover $(U_{j})_{j\in\mathbb{I}}$ of $X$ such that
          the sequence
          \[
          \begin{tikzcd}
            \cdots \arrow{r} & \mathscr{F}_{i-1}\mid_{U_{j}}\arrow{r}
            & \mathscr{F}_{i}\mid_{U_{j}}\arrow{r}
            & \mathscr{F}_{i+1}\mid_{U_{j}}\arrow{r} & \cdots
          \end{tikzcd}
          \]
          is exact for every $j\in I$.
    \item For every $P\in X$, the induced sequence
          \[
          \begin{tikzcd}
            \cdots \arrow{r} & (\mathscr{F}_{i-1})_{P}\arrow{r}
            & (\mathscr{F}_{i})_{P}\arrow{r}
            & (\mathscr{F}_{i+1})_{P}\arrow{r} & \cdots
          \end{tikzcd}
          \]
          is exact.
  \end{enumerate}
\end{thm}

\section{Sheaf cohomology}
Studying sheaves using homological algebra turns out to be surprisingly
useful in many situations. For example, knowing that there is a SES
\[
  \begin{tikzcd}
    0 \arrow{r} & \mathscr{F} \arrow{r} & \mathscr{G} \arrow{r} &
    \mathscr{H} \arrow{r} & 0
  \end{tikzcd}
\]
lets us relate the three sheaves together. By Thm.~\ref{thm:ses_equivalence}
this information is inherently \emph{local} since this sequence is exact
if and only if the corresponding sequences on stalks are exact.
Then the question is: Can we get \emph{global} information from
such exact sequences? We would hope that just as the sequence is exact
on stalks, it would also be exact on global sections:
\[
\begin{tikzcd}
  0 \arrow{r} & \Gamma(\mathscr{F}) \arrow{r} & \Gamma(\mathscr{G})
  \arrow{r} & \Gamma(\mathscr{H}) \arrow{r} & 0
\end{tikzcd}
\]
Unfortunately, this is not the case. One can prove that this sequence
is exact at $\Gamma(\mathscr{F})$ and $\Gamma(\mathscr{G})$, but it
is not always exact at $\Gamma(\mathscr{H})$.
\begin{cat}
  We say that the global sections functor $\Gamma: \text{Sh}(X)\to \textbf{Ab}$ is left-exact but not right-exact.
\end{cat}
In more concrete terms, if we know that a morphism $f: \mathscr{F}
\to\mathscr{G}$ is injective on stalks, it is also injective on
global sections. But if the morphism is surjective on stalks, we don't
know whether or not it is surjective on global sections.
% TODO: Add an example

Although exact sequences are not completely preserved under taking
global sections, we won't give up! There might still be \emph{a way of
measuring how much exactness fails}. We could measure the
\emph{obstruction} to exactness by continuing the sequence to the right
so that the following sequence is exact.
\[
\begin{tikzcd}
  0 \arrow{r} & \Gamma(\mathscr{F}) \arrow{r} & \Gamma(\mathscr{G})
  \arrow{r} & \Gamma(\mathscr{H})
  \arrow[out=0, in=180, looseness=1.5, overlay]{dll} & \\
    & H^{1}(\mathscr{F}) \arrow{r} & H^{1}(\mathscr{G})
  \arrow{r} & H^{1}(\mathscr{H}) \arrow{r} & \cdots \\
  \cdots \arrow{r}& H^{i}(\mathscr{F}) \arrow{r} & H^{i}(\mathscr{G})
  \arrow{r} & H^{i}(\mathscr{H}) \arrow{r} & \cdots
\end{tikzcd}
\]
This problem of extending incomplete short exact sequences appears
elsewhere in homological algebra and is generally solved by constructing
so-called \emph{derived functors}. These modules $H^{i}(-)$ given by derived
functors are then called the sheaf cohomology groups. In practise, they are
difficult to compute, so we want to find an alternative definition,
which is easier to work with. This is achieved by \emph{\v Cech cohomology}.
Next, I will explain derived functors and show how we arrived at \v Cech
cohomology from there. However, this will involve a lot of new, high-level
concepts and category theory. Understanding the philosophy behind
\v Cech cohomolgy is not essential, and one can safely skip straight to the
definition of the \v Cech cohomology groups given in Def.~\ref{def:cech}.

\subsection{Motivating \v Cech cohomology}

\section{The Riemann-Roch theorem}
Equiped with sheaf cohomology, we will prove the Riemann-Roch theorem
using the methods we have learnt. But first we need to quickly review
two constructions needed to state the Riemann-Roch theorem: \emph{divisors}
and \emph{differentials}.

\subsection{Divisors and differentials}

\subsection{Proof of Riemann-Roch}
We are now able to state and prove an incomplete version of the Riemann-Roch
theorem, which we will make complete after proving Serre duality.

\begin{lnote}
  In this section, $X$ will always denote an irreducible, non-singular,
  complete algebraic curve, and any divisor $D$ is defined on $X$.
\end{lnote}

\begin{thm}[Riemann-Roch, incomplete version]
  \label{thm:riemann_roch_incomplete}
  For every divisor $D$,
  \[
    h^{0}(X, \mathcal{L}(D))-h^{1}(X, \mathcal{L}(D))=\deg(D)+1-g,
  \]
  where $g=h^{1}(X, \mathcal{O}_{X})$.
\end{thm}
\begin{proof}
  We can use an induction argument, because any divisor $D$ can be
  obtained from the zero divisor by adding and subtracting points.
  Thus, we proceed by first proving the base case and then proving
  the induction step.

  \begin{description}[style=nextline]
    \item[base case$\big)$]
          Since $\mathcal{L}(0)=\mathscr{O}_{X}$ and $\deg(0)=0$,
          we need to verify that
          \[h^{0}(X, \mathscr{O}_{X})-h^{1}(X, \mathscr{O}_{X})=1-g.\]
          But note that the only globally defined regular functions
          on $X$ are constant by Cor.~\ref{cor:global_const},
          and thus they form a one-dimensional vector space.
          Moreover, $h^{1}(X, \mathscr{O}_{X})=g$ by definition
          so that the equality holds.
    \item[induction step$\big)$]
          In the induction step we want to relate the zeroth and first
          cohomology groups of $\mathcal{L}(D)$ to the zeroth and
          first cohomology groups of $\mathcal{L}(D+P)$, where
          $P$ is some point. To do this, we first note that
          $\mathcal{L}(D)$ is a subsheaf of $\mathcal{L}(D+P)$, since
          the orders of germs of $\mathcal{L}(D+P)$ at $P$ are allowed to
          be smaller than the orders of germs of $\mathcal{L}(D)$ at $P$.
          Thus, there is an exact sequence
          \[
          \begin{tikzcd}
            0\arrow{r} & \mathcal{L}(D)\arrow{r} & \mathcal{L}(D+P)\arrow{r}
            & Q\arrow{r} & 0
          \end{tikzcd}
          \]
          where $Q$ is the quotient sheaf. The stalks of $Q$ are clearly
          zero away from $P$. The stalk at $P$ consists of zero and
          elements of the form $u/t^{n+1}$, where $t$ is the local
          uniformiser of $\mathscr{O}_{P}$, $u$ is a unit in
          $\mathcal{O}_{P}$, and $n$ is the order of $P$ in $D$.
          As $\mathcal{O}_{P}/(t)=k$, we can write $u=vt+r$,
          where $v\in\mathcal{O}_{P}$ and $r\in k$.
          Then,
          \[\frac{u}{t^{n+1}}=\frac{v}{t^n}+\frac{r}{t^{n+1}}\]
          Since $v/t^n$ is an element of $\mathcal{L}(D)_{P}$, we conclude
          that every element of $\mathcal{L}(D+P)_{P}$ is equivalent to
          an element $r/t^{n+1}$ modulo $\mathcal{L}(D)_{P}$ for some
          $r\in k$. Therefore, $Q_{P}\cong k$ and $Q$ is the skyscraper sheaf!

          Now we apply our cohomology machinery on the SES
          \[
          \begin{tikzcd}
            0\arrow{r} & \mathcal{L}(D)\arrow{r} & \mathcal{L}(D+P)\arrow{r}
            & k_{P}\arrow{r} & 0
          \end{tikzcd}
          \]
          to get the following exact sequence (using
          Prop.~\ref{prop:sky_cohom}).
          \[
          \begin{tikzcd}
            0\arrow{r} & H^{0}(X, \mathcal{L}(D))\arrow{r}
            & H^{0}(X, \mathcal{L}(D+P))\arrow{r}
            & H^{0}(X, k_{P}) \\
            \arrow{r} & H^{1}(X, \mathcal{L}(D))\arrow{r}
            & H^{1}(X, \mathcal{L}(D+P))\arrow{r} & 0
          \end{tikzcd}
          \]
          Now, by Prop.~\ref{prop:homology_dim},
          \[
          h^{0}(X,\mathcal{L}(D))-h^{0}(X, \mathcal{L}(D+P))
          +1-h^{1}(X,\mathcal{L}(D))+h^{1}(X,\mathcal{L}(D+P)) = 0
          \]
          Therefore,
          \begin{align*}
            h^{0}(X,\mathcal{L}(D+P))&-h^{1}(X,\mathcal{L}(D+P))
            =\left(h^{0}(X,\mathcal{L}(D))-h^{1}(X,\mathcal{L}(D+P))\right)
              +1 \\
            =&\deg(D)+1-g+1\quad\text{(by induction hypothesis)} \\
            =&\deg(D+P)+1-g
          \end{align*}
          This is exactly the induction step we wanted to prove.
          We also need to prove
          \[
            h^{0}(X,\mathcal{L}(D-P))-h^{1}(X,\mathcal{L}(D-P))
            =\deg(D-P)+1-g
          \]
          but we can run the same argument starting with the SES
          \[
          \begin{tikzcd}
            0\arrow{r} & \mathcal{L}(D-P)\arrow{r} & \mathcal{L}(D)\arrow{r}
            & k_{P}\arrow{r} & 0
          \end{tikzcd}
          \]
  \end{description}
\end{proof}

This form of the theorem is not the most useful one, because computing
$h^{1}(X,\mathcal{L}(D))$ is not easy. We will devote the last part
of the paper to proving the Serre duality theorem, which relates
zeroth and first cohomology groups. The theorem implies the
following equality of dimensions:
\[h^{1}(X,\mathcal{L}(D))=h^{0}(X,\mathcal{L}(K_{X}-D))\]

Now, this equality lets us write the complete form of the Riemann-Roch
theorem.
\begin{thm}[Riemann-Roch, complete version]\label{thm:riemann_roch_complete}
  For every divisor $D$,
  \[
    h^{0}(X, \mathcal{L}(D))-h^{0}(X, \mathcal{L}(K_{X}-D))=\deg(D)+1-g,
  \]
  where $g=h^{1}(X, \mathcal{O}_{X})$.
\end{thm}

\subsection{Applications}
Before proving the Serre duality theorem, I want to take some time to
look at applications of the Riemann-Roch theorem.

\section{Serre duality}

\section{Bibliography}
\bibliographystyle{alpha}
\bibliography{refs}

\end{document}
