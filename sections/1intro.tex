\section{Introduction}
In this paper I will give a proof of the Riemann-Roch theorem
by using one of the most important tools in modern algebraic geometry ---
sheaf cohomology. This text is aimed at the student with
basic knowledge of varieties, and will function as an introduction
to the topic of sheaf cohomolgy, motivated by the problem of classifying
algebraic curves.

% TODO: Insert a short description of Riemann-Roch

The main ingredient in the proof of the Riemann-Roch theorem is
Serre duality, which is a more general result. The proof will again
be cohomological and is usually rather abstract. However, I will
approach the proof by giving concrete interpretations of
the cohomology groups by following Serre's exposition in \cite{serre}.

The reader is assumed to be familiar with basic algebraic geometry.
An excellent introduction to the topic is \cite{reid}, and a good set
of lecture notes which goes beyond the basics is \cite{gathmann}.
Through out the paper, I will also try to point out category theoretical
contexts of the concepts discussed. However, it is not necessary to know
any category theory to understand this paper, and I will always place a
warning for people who are not fond of abstract nonsense.
