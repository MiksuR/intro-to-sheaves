\section*{Introduction}
\addcontentsline{toc}{section}{Introduction}
In this article, I am introducing sheaf theory and the cohomology of
sheaves in the context of algebraic geometry to prove the Riemann-Roch 
theorem, which is one of the most important theorems in the classification
of algebraic curves. This text is aimed at undergraduate students with
basic knowledge of varieties.

Sheaves are formed by attaching algebraic data to \emph{local} patches of 
a space. We usually associate data about rational functions, and 
the Riemann-Roch theorem will give \emph{global} information about certain 
types of functions on a space. I will use the theorem to show the existence 
of a globally defined function on a curve $X$ of genus 0, which defines 
an isomorphism with the projective line. Thus, we get a complete 
classification of curves of genus 0. One can use the Riemann-Roch theorem 
to classify curves of higher genera as well.

The main ingredient in the proof of the Riemann-Roch theorem is Serre
Duality, which is a more general result. The proof will again
be cohomological and is usually rather abstract. However, I will
approach the proof by giving concrete interpretations of
the cohomology groups by following Serre's exposition \cite{serre}.

The reader is assumed to know basic algebraic geometry. An excellent
introduction to the topic is \cite{reid}, and a good set of lecture notes
which goes beyond the basics is \cite{gathmann}. I will use the language
of discrete valuation rings, which is not explained in this paper. See
\cite{fulton} for an introduction to this topic. Having familiarity with 
the basic constructions of homological algebra is desirable. Throughout 
the paper, I will also try to point out category theoretical contexts of
the concepts discussed. However, it is not necessary to know any category
theory to understand this paper, and I will always place a warning for
people who are not fond of abstract nonsense.
