\section{Sheaf cohomology}
Studying sheaves using homological algebra turns out to be surprisingly
useful in many situations. For example, knowing that there is a SES
\[
  \begin{tikzcd}
    0 \rar & \mathscr{F} \rar & \mathscr{G} \rar &
    \mathscr{H} \rar & 0
  \end{tikzcd}
\]
lets us relate the three sheaves together. By Thm.~\ref{thm:ses_equivalence}
this information is inherently \emph{local} since this sequence is exact
if and only if the corresponding sequences on stalks are exact.
Then the question is: Can we get \emph{global} information from
such exact sequences? We would hope that just as the sequence is exact
on stalks, it would also be exact on global sections:
\[
\begin{tikzcd}
  0 \rar & \Gamma(\mathscr{F}) \rar & \Gamma(\mathscr{G})
  \rar & \Gamma(\mathscr{H}) \rar & 0.
\end{tikzcd}
\]
Unfortunately, this is not the case. For example, let
$X=\mathbb{P}^{1}_{\mathbb{C}}$ and consider the sheaf morphism
\[f:\mathscr{O}_{X}\to\mathbb{C}_{P_{0}}\oplus\mathbb{C}_{P_{1}},\]
which evaluates $f$ at some points $P_{0},P_{1}\in X$. Then, the
morphism is clearly surjective on the stalks. But it is not surjective
on global sections, since the global sections of $\mathscr{O}_{X}$ are
the constant functions. In other words, the exact sequence
\[\mathscr{O}_{X}\to\mathbb{C}_{P_{0}}\oplus\mathbb{C}_{P_{1}}\to 0\]
does not give an exact sequence on global sections.

However, we have the following.
\begin{prop}
  If the sequence
  \[
  \begin{tikzcd}
    0\rar & \mathscr{F}\rar & \mathscr{G}\rar & \mathscr{H}\rar & 0
  \end{tikzcd}
  \]
  is exact, then the sequence
  \[
  \begin{tikzcd}
    0\rar & \Gamma(\mathscr{F})\rar{\alpha} & \Gamma(\mathscr{G})\rar{\beta}
    & \Gamma(\mathscr{H})
  \end{tikzcd}
  \]
  is also exact.
\end{prop}
\begin{proof}\hfill
  \begin{description}[style=nextline]
    \item[Exactness at $\Gamma(\mathscr{F})$]
          Suppose $\phi\in\ker(\alpha)$ % TODO
  \end{description}

\end{proof}
\begin{cat}
  We say that the global sections functor $\Gamma: \text{Sh}(X)\to \textbf{Ab}$ is left-exact but not right-exact.
\end{cat}

Although exact sequences are not completely preserved under taking
global sections, we won't give up! There might still be \emph{a way of
measuring how much exactness fails}. We could measure the
\emph{obstruction} to exactness by continuing the sequence to the right
so that the following sequence is exact.
\[
\begin{tikzcd}
  0 \rar & \Gamma(\mathscr{F}) \rar & \Gamma(\mathscr{G})
  \rar\dar[phantom, ""{coordinate, name=Z}] & \Gamma(\mathscr{H})
  \arrow[rounded corners, to path={ -- ([xshift=2ex]\tikztostart.east)
    |- (Z) -| ([xshift=-2ex]\tikztotarget.west) -- (\tikztotarget)},
  overlay]{dll} & \\
    & H^{1}(\mathscr{F}) \rar & H^{1}(\mathscr{G})
  \rar & H^{1}(\mathscr{H}) \rar & \cdots \\
  \cdots \rar& H^{i}(\mathscr{F}) \rar & H^{i}(\mathscr{G})
  \rar & H^{i}(\mathscr{H}) \rar & \cdots.
\end{tikzcd}
\]
This problem of extending incomplete short exact sequences appears
elsewhere in homological algebra and is generally solved by constructing
so-called \emph{derived functors}. These vector spaces $H^{i}(-)$ given by
derived functors are then called the sheaf cohomology groups. In practise,
they are difficult to compute, and thus I will define the \emph{\v Cech
  cohomology} which is a tool for computing sheaf cohomology.
In the next subsection I will introduce derived functors and deduce the
definition of \v Cech cohomology. The contents of the subsection will be
more technical than the rest of the paper, and it is probably a good idea
to skip straight to Definition~\ref{def:cech}, which can be taken as
\emph{the} definition of sheaf cohomology.

\subsection{From derived functors to \v Cech cohomology*}
Let us consider the general problem of extending a left-exact
functor $F:\mathcal{A}\to\mathcal{B}$ between abelian categories
(which one may think of as categories of modules) to the right. Thus, for
objects $A,B,C$ of $\mathcal{A}$ fitting into a SES
\[\begin{tikzcd}
    0\rar & A\rar & B\rar & C\rar & 0
  \end{tikzcd},\]
we want to find functors $R^{i}F:\mathcal{A}\to\mathcal{B}$ and
connecting morphisms so that the sequence
\[
\begin{tikzcd}
  0 \rar & F(A) \rar & F(B)
  \rar\dar[phantom, ""{coordinate, name=Z}] & F(C)
  \arrow[rounded corners, to path={ -- ([xshift=2ex]\tikztostart.east)
    |- (Z) -| ([xshift=-2ex]\tikztotarget.west) -- (\tikztotarget)},
  overlay]{dll} & \\
  & R^{1}F(A)\rar & R^{1}F(B)\rar
  & R^{1}F(C)\rar & \cdots \\
  \cdots\rar & R^{i}F(A)\rar & R^{i}F(B)\rar
  & R^{i}F(C)\rar & \cdots.
\end{tikzcd}
\]
is exact. The functors $R^{i}F$ will be called the \emph{right
  derived functors} of $F$. The following lemma from homological
algebra gives a hint of what approach we should take to find such functors.
\begin{lemm}[Zig-zag lemma]
  Suppose $A^{\bullet},B^{\bullet},C^{\bullet}$ are cochain complexes in some
  abelian category. If there is a SES
  \[\begin{tikzcd}
      0\rar & A^{\bullet}\rar & B^{\bullet}\rar & C^{\bullet}\rar & 0
    \end{tikzcd},\]
  then there are maps between the cohomology groups of these complexes
  such that the sequence
  \[\begin{tikzcd}
    & H^{0}(A^{\bullet}) \rar & H^{0}(B^{\bullet})
    \rar\dar[phantom, ""{coordinate, name=Z}] & H^{0}(C^{\bullet})
    \arrow[rounded corners, to path={ -- ([xshift=2ex]\tikztostart.east)
      |- (Z) -| ([xshift=-2ex]\tikztotarget.west) -- (\tikztotarget)},
    overlay]{dll} & \\
    & H^{1}(A^{\bullet}) \rar & H^{1}(B^{\bullet})\rar
    & H^{1}(C^{\bullet}) \rar & \cdots \\
    \cdots \rar& H^{i}(A^{\bullet}) \rar & H^{i}(B^{\bullet})\rar
    & H^{i}(C^{\bullet}) \rar & \cdots
    \end{tikzcd}\]
  is exact.
\end{lemm}
\begin{proof}
  The maps $H^{i}(A^{\bullet})\to H^{i}(B^{\bullet})$ and $H^{i}(B^{\bullet})
  \to H^{i}(C^{\bullet})$ are given by functoriality, and the connecting
  morphisms $H^{i}(C^{\bullet})\to H^{i+1}(A^{\bullet})$ are given by the
  snake lemma. Working out the details of this diagram chasing argument
  is a good exercise for the reader. I will instead use a spectral sequence
  argument (See below for more discussion on spectral sequences).

  Define the zeroth page of a spectral sequence to be the following
  double complex given by the SES of complexes.
  \[\begin{tikzcd}
      & \vdots & \vdots & \vdots & \\
      0\rar & A^{2}\rar\uar & B^{2}\rar\uar & C^{2}\rar\uar & 0 \\
      0\rar & A^{1}\rar\uar & B^{1}\rar\uar & C^{1}\rar\uar & 0 \\
      0\rar & A^{0}\rar\uar & B^{0}\rar\uar & C^{0}\rar\uar & 0 \\
      & 0\uar & 0\uar & 0\uar &
    \end{tikzcd}\]
  Since the rows are exact the first page is zero when we use the rightward
  orientation. Now, let us compute the first page using upward orientation.
  We get the following.
  \[\begin{tikzcd}
      & \vdots & \vdots & \vdots & \\
      0\rar & H^{2}(A^{\bullet})\rar{\alpha_{2}}
      & H^{2}(B^{\bullet})\rar{\beta_{2}} & H^{2}(C^{\bullet})\rar & 0 \\
      0\rar & H^{1}(A^{\bullet})\rar{\alpha_{1}}
      & H^{1}(B^{\bullet})\rar{\beta_{1}} & H^{1}(C^{\bullet})\rar & 0 \\
      0\rar & H^{0}(A^{\bullet})\rar{\alpha_{0}}
      & H^{0}(B^{\bullet})\rar{\beta_{0}} & H^{0}(C^{\bullet})\rar & 0 \\
    \end{tikzcd}\]
  Finally, in the second page we will have

  \begin{center}
  \begin{tikzpicture}[commutative diagrams/every diagram, x=2.2cm, y=1.7cm]
    \clip (0.8,0.5) rectangle (5.2,3.5);
    \node (A0) at (2,1) {$\ker(\alpha_{0})$};
    \node (B0) at (3,1) {$\frac{\ker(\beta_{0})}{\im(\alpha_{0})}$};
    \node (C0) at (4,1) {$\coker(\beta_{0})$};
    \node (A1) at (2,2) {$\ker(\alpha_{1})$};
    \node (B1) at (3,2) {$\frac{\ker(\beta_{1})}{\im(\alpha_{1})}$};
    \node (C1) at (4,2) {$\coker(\beta_{1})$};
    \node (A2) at (2,3) {$\ker(\alpha_{2})$};
    \node (B2) at (3,3) {$\frac{\ker(\beta_{2})}{\im(\alpha_{2})}$};
    \node (C2) at (4,3) {$\coker(\beta_{2})$};
    \node (lu) at (1,3) {$0$};
    \node (lm) at (1,2) {$0$};
    \node (rd) at (5,1) {$0$};
    \node (rm) at (5,2) {$0$};

    \path[commutative diagrams/.cd, every arrow, every label]
      (0,4) edge (A2)
      (0,3) edge (A1)
      (0,2) edge (A0)
      (lu) edge (B1)
      (lm) edge (B0)
      (1,4) edge (B2)
      (2,4) edge (C2)
      (A2) edge (C1)
      (A1) edge (C0)
      (B2) edge (rm)
      (B1) edge (rd)
      (C2) edge (6,2)
      (C1) edge (6,1)
      (C0) edge (6,0)
      (A0) edge (4,0)
      (B0) edge (5,0);
  \end{tikzpicture}
  \end{center}
  One can see that the spectral sequence will converge on the third
  page. Since the sequence converges to zero, the sequences
  \[\begin{tikzcd}
      0\rar & \ker(\alpha_{i+1})\rar & \coker(\beta_{i})\rar & 0,
    \end{tikzcd}\]
  given by the differentials on the second page must be exact.
  These isomorphisms induce maps
  \[\delta_{i}:H^{i}(C^{\bullet})\to H^{i+1}(A^{\bullet}).\]
  The convergence of the spectral sequence also implies that
  $\ker(\beta_{i})/\im(\alpha_{i})=0$. Putting these results together,
  we see that the sequence
  \[\begin{tikzcd}
    & H^{0}(A^{\bullet})\rar{\alpha_{0}} & H^{0}(B^{\bullet})
    \rar{\beta_{0}}\dar[phantom, ""{coordinate, name=Z}] & H^{0}(C^{\bullet})
    \arrow[rounded corners, to path={[pos=0] --
      ([xshift=2ex]\tikztostart.east) |- (Z) -|
      ([xshift=-2ex]\tikztotarget.west)\tikztonodes -- (\tikztotarget)},
    "\delta_{0}"']{dll} & \\
    & H^{1}(A^{\bullet})\rar{\alpha_{1}} & H^{1}(B^{\bullet})\rar{\beta_{1}}
    & H^{1}(C^{\bullet})\rar{\delta_{1}} & \cdots \\
    \cdots \rar{\delta_{i-1}}& H^{i}(A^{\bullet})\rar{\alpha_{i}}
    & H^{i}(B^{\bullet})\rar{\beta_{i}} & H^{i}(C^{\bullet})\rar{\delta_{i}}
    & \cdots
    \end{tikzcd}\]
  is exact.
\end{proof}

Therefore, in order to extend the sequence
\[\begin{tikzcd}
    0\rar & F(A)\rar & F(B)\rar & F(C)
  \end{tikzcd},\]
we wish to find cocomplexes $A^{\bullet}, B^{\bullet}, C^{\bullet}$
associated to $A, B, C$ such that
\begin{enumerate}
  \item The cochain complexes $A^{\bullet}, B^{\bullet}, C^{\bullet}$ fit into a
        SES
        \[\begin{tikzcd}
            0\rar & A^{\bullet}\rar & B^{\bullet}\rar & C^{\bullet}\rar & 0
          \end{tikzcd}\]
  \item The zeroth cohomology coincides with $F$:
        \[H^{0}(A^{\bullet})=F(A),
        \quad H^{0}(B^{\bullet})=F(B),
        \quad H^{0}(C^{\bullet})=F(C).\]
\end{enumerate}

One way of associating a cochain complex to an object $A$ is to take its
\emph{resolution}. In other words, by finding objects $A^{i}$ and
morphisms such that the sequence
\[\begin{tikzcd}
    0\rar & A\rar{\alpha} & A^{0}\rar{\alpha_{0}} & A^{1}\rar{\alpha_{1}}
    & A^{2}\rar{\alpha_{2}} & \cdots
  \end{tikzcd}\]
is exact. Let us concentrate on the first few terms of this sequence.
\[\begin{tikzcd}
    0\rar & A\rar{\alpha} & A^{0}\rar{\alpha_{0}} & A^{1}
  \end{tikzcd}.\]
Since $F$ is left-exact, we get an another exact sequence:
\[\begin{tikzcd}
    0\rar & F(A)\rar{\alpha^{\ast}}
    & F(A^{0})\rar{\alpha_{0}^{\ast}} & F(A^{1})
  \end{tikzcd}.\]
Now, exactness implies that $\ker(\alpha_{0}^{\ast})=\im(\alpha^{\ast})
=F(A)$. Therefore, if I were to replace $F(A)$ by $0$,
then the cohomology at $F(A^{0})$ would be $F(A)$!
Thus, considering the cochain complex
\[\begin{tikzcd}
    0\rar & F(A^{0})\rar{\alpha_{0}^{\ast}}
    & F(A^{1})\rar{\alpha_{1}^{\ast}}
    & F(A^{2})\rar{\alpha_{2}^{\ast}} & \cdots
  \end{tikzcd},\]
we see that taking the cohomology of this complex will give
\[H^{0}(F(A^{\bullet}))=F(A).\]
Hence, if I construct the cochain complexes $A^{\bullet},B^{\bullet},
C^{\bullet}$ from resolutions of $A,B,C$ as above, then the resulting
cohomolgy will satisfy the second requirement. Next, I want to find the right
type of resolution so that the cochain complexes satisfy the first
requirement above.

What we have currently is the following picture.
\[\begin{tikzcd}
    & \vdots & \vdots & \vdots & \\
    & A^{1}\uar{\alpha_{1}} & B^{1}\uar{\beta_{1}} & C^{1}\uar{\gamma_{1}} & \\
    & A^{0}\uar{\alpha_{0}} & B^{0}\uar{\beta_{0}} & C^{0}\uar{\gamma_{0}} & \\
    0\rar& A\rar{f}\uar{\alpha} & B\rar{g}\uar{\beta} & C\rar\uar{\gamma}&0\\
    & 0\uar & 0\uar & 0\uar &
  \end{tikzcd}\]
If I use \emph{injective resolutions}, the diagram can be filled in with
approprate morphisms so that it gives a SES of complexes.
\begin{defin}
  An object $I$ of an abelian category $\mathcal{A}$ is injective,
  if for every injection $f:A\hookrightarrow B$ and every morphism
  $g:A\to I$, there is a morphism $B\to I$ such that the following
  diagram commutes.
  % TODO: Orient the diagram so that it lines up with the diagram
  %       in the proof of Prop. 3.4.
  \[\begin{tikzcd}
      B\rar[dashed] & I \\
      A\uar[hook]{f}\arrow["g"']{ur}
    \end{tikzcd}\]
\end{defin}
Then, an injective resolution of an object $A$ is a long exact sequence
\[\begin{tikzcd}
    0\rar & A\rar & I^{0}\rar & I^{1}\rar & \cdots
  \end{tikzcd}\]
where the $I^{i}$ are injective. Note that it is not obvious that an object
of an abelian category should have an injective resolution. Thus, we assume
that the category $\mathcal{A}$ has \emph{enough injectives} meaning that
for every object $A$ of $\mathcal{A}$, there is an injection
$A\hookrightarrow I$ into some injective object $I$. Then, for every object
$A$ we can construct an injective resolution inductively. The first object
$I^{0}$ is given directly by the assumption. Then suppose we have constructed
an exact sequence
\[\begin{tikzcd}
    0\rar & A\rar & I^{0}\rar{\iota_{0}} & \cdots\rar
    & I^{n-1}\rar{\iota_{n-1}} & I^{n}
  \end{tikzcd}.\]
Let us take $I^{n+1}$ to be an injective object such that there is
an injection $\coker(\iota_{n-1})\hookrightarrow I^{n+1}$. Then we have
\[\begin{tikzcd}
    0\rar & A\rar & I^{0}\rar{\iota_{0}} & \cdots\rar
    & I^{n-1}\rar{\iota_{n-1}} & I^{n}\rar[two heads]
    & \coker(\iota_{n-1})\rar[hook] & I^{n+1}
  \end{tikzcd}.\]
When the injection is composed with the projection, we get the exact sequence
\[\begin{tikzcd}
    0\rar & A\rar & I^{0}\rar{\iota_{0}} & \cdots\rar
    & I^{n-1}\rar{\iota_{n-1}} & I^{n}\rar{\iota_{n}} & I^{n+1}
  \end{tikzcd}.\]

Now, it follows directly from the definitions that any SES can
be lifted along an injective resolution:
\begin{prop}
  If $A, B, C$ are objects of an abelian category $\mathcal{A}$ with enough
  injectives fitting into a SES
  \[\begin{tikzcd}
      0\rar & A\rar{f} & B\rar{g} & C\rar & 0
    \end{tikzcd},\]
  and $A$ and $C$ have injective resolutions $A^{\bullet}$ and $C^{\bullet}$,
  then there is an injective resolution $B^{\bullet}$ of $B$ and maps
  $f^{i}:A^{i}\to B^{i}$ and $g^{i}:B^{i}\to C^{i}$ such that the following
  commutative diagram has exact rows.
\[\begin{tikzcd}
    & \vdots & \vdots & \vdots & \\
    0\rar & A^{1}\rar{f^{1}}\uar{\alpha_{1}} & B^{1}\rar{g^{1}}\uar{\beta_{1}}
    & C^{1}\rar\uar{\gamma_{1}} & 0 \\
    0\rar & A^{0}\rar{f^{0}}\uar{\alpha_{0}} & B^{0}\rar{g^{0}}\uar{\beta_{0}}
    & C^{0}\rar\uar{\gamma_{0}} & 0 \\
    0\rar& A\rar{f}\uar{\alpha} & B\rar{g}\uar{\beta} & C\rar\uar{\gamma}&0\\
    & 0\uar & 0\uar & 0\uar &
  \end{tikzcd}\]
\end{prop}
\begin{proof}
  TODO.
\end{proof}
Now I only need to apply the functor $F$ on this double complex
and remove the bottom row. The only worry is that the rows don't stay
exact. Thus, I will need to prove one more small result.
\begin{prop}
  A left-exact functor $F:\mathcal{A}\to\mathcal{B}$ between
  abelian categories is exact on short exact sequences of injective objects.
\end{prop}
\begin{proof}
  Suppose $I_{1}, I_{2}, I_{3}$ are injective objects in $\mathcal{A}$
  such that there is a SES
  \[\begin{tikzcd}
      0\rar & I_{1}\rar & I_{2}\rar & I_{3}\rar & 0
    \end{tikzcd}.\]

  TODO.
\end{proof}

In summary, given a left-exact functor $F:\mathcal{A}\to
\mathcal{B}$ and an object $A$ of $\mathcal{A}$, one constructs the $i$th
right derived functor of $F$ at $A$ in the following way:
\begin{enumerate}
  \item Find an injective resolution $0\to A\to I^{\bullet}$ of $A$
  \item Apply $F$ on the cochain complex $0\to I^{\bullet}$
  \item Take the $i$th cohomology of this cochain complex
        \[\begin{tikzcd}
            0\rar & F(I^{0})\rar & F(I^{1})\rar
            & F(I^{2})\rar & \cdots
          \end{tikzcd}\]
\end{enumerate}
We denote $R^{i}F(A)=H^{i}(F(I^{\bullet}))$ for the
derived functor.
\begin{bcat}
  One could ask: ``How do we know that derived functors give the `right'
  way of extending the left-exact functor to the right?'' This can be
  formalised by considering so-called (cohomological) $\delta$-functors,
  which consist of pairs $(T^{i},\delta^{i})$, where
  \begin{enumerate}
    \item The $T^{i}:\mathcal{A}\to\mathcal{B}$ are functors between abelian
          categories with $T^{i}=0$ for $i<0$, and
    \item $\delta^{i}:T^{i}(C)\to T^{i+1}(A)$ are morphisms in $\mathcal{B}$,
  \end{enumerate}
  such that for every SES
  \[\begin{tikzcd}
      0\rar & A\rar & B\rar & C\rar & 0
    \end{tikzcd}\]
  in $\mathcal{A}$, we have a long exact sequence
  \[\begin{tikzcd}
      0\rar & T^{0}(A)\rar & T^{0}(B)\rar\dar[phantom, ""{coordinate, name=Z}]
      & T^{0}(C)\arrow[rounded corners, to path={[pos=0] --
      ([xshift=2ex]\tikztostart.east) |- (Z) -|
      ([xshift=-2ex]\tikztotarget.west)\tikztonodes -- (\tikztotarget)},
    "\delta_{0}"']{dll} & \\
    & T^{1}(A)\rar & T^{1}(B)\rar & T^{1}(C)\rar{\delta_{1}} & \cdots \\
    \cdots \rar{\delta_{i-1}}& T^{i}(A)\rar & T^{i}(B)\rar
    & T^{i}(C)\rar{\delta_{i}} & \cdots
    \end{tikzcd}\]
  In addition, we require functoriality of this construction: If
  \[\begin{tikzcd}
      0\rar & A\rar\dar & B\rar\dar & C\rar\dar & 0 \\
      0\rar & A^{\prime}\rar & B^{\prime}\rar & C^{\prime}\rar & 0
    \end{tikzcd}\]
  is a morphism of short exact sequences in $\mathcal{A}$, then the
  squares
  \[\begin{tikzcd}
      T^{i}(C)\rar{\delta^{i}}\dar & T^{i+1}(A)\dar \\
      T^{i}(C^{\prime})\arrow["\delta^{i}"']{r} & T^{i+1}(A^{\prime})
    \end{tikzcd}\]
  commute.

  One can then define morphisms of $\delta$-functors so that they form
  a category. Then, one can define the concept of a \emph{universal}
  $\delta$\emph{-functor} in this category: A $\delta$-functor
  $(T^{i},\delta^{i})$ is universal if for every other $\delta$-functor
  $(S^{i},\gamma^{i})$ with a natural transformation
  $\alpha: T^{0}\Rightarrow S^{0}$, there is a unique morphism of
  $\delta$-functors $(T^{i},\delta^{i})\to (S^{i},\gamma^{i})$ extending
  $\alpha$. Then, one can prove that derived functors define a universal
  $\delta$-functor. I will not spell out the details here, but further
  information on $\delta$-functors can be found in \cite{vakil}.
\end{bcat}
Now, consider an $\mathcal{O}_{X}$-module $\mathcal{F}$ on a space $X$.
Its \emph{sheaf cohomology} is then defined as the right derived functor of
the global sections functor:
\[
  H^{i}(X, \mathcal{F}) := R^{i}\Gamma.
\]
This definition relies on the assumption that the category of sheaves of
$\mathcal{O}_{X}$-modules has enough injectives.
\begin{prop}
  The category $\mathcal{O}_{X}$\textup{\textbf{-Mod}} has enough injectives.
\end{prop}
\begin{proof}
  TODO.
\end{proof}
Derived functors provide a good framework for the cohomology theory of
sheaves, but working with injective resolutions is difficult in practise.
% TODO: Expand on reason behind this
Therefore, we need an alternative way of constructing the cohomology groups.

I will do this by replacing injective resolutions by \emph{acyclic
  resolutions}. If $F:\mathcal{A}\to\mathcal{B}$ is a left-exact
functor between abelian categories, then an object $A$ of $\mathcal{A}$ is
acyclic if $R^{i}F(A)=0$ for $i>0$. I will justify this in the
next proposition.
\begin{prop}
  Let $F:\mathcal{A}\to\mathcal{B}$ be a left-exact functor
  between abelian categories, where $\mathcal{A}$ has enough injectives.
  Suppose $A$ is an object of $\mathcal{A}$ with acyclic resolution
  \[\begin{tikzcd}
      0\rar & A\rar & A^{0}\rar & A^{1}\rar & \cdots
    \end{tikzcd}.\]
  Then, computing the cohomology of the cochain complex
  \[\begin{tikzcd}
      0\rar & F(A^{0})\rar & F(A^{1})\rar & \cdots
    \end{tikzcd}\]
  agrees with the right derived functor of $F$ at $A$.
\end{prop}
\begin{proof}
  TODO.
\end{proof}
Next, I want to find an acyclic resolution for a given sheaf so that I can
calculate its sheaf cohomology. First I will make the following observation.
\begin{prop}\label{prop:affine_cohom_vanishes}
  If $X$ is an affine variety and $\mathcal{F}$ is a sheaf of
  $\mathcal{O}_{X}$-modules, then $H^{i}(X, \mathcal{F})=0$ for all $i>0$.
\end{prop}
\begin{proof}
  TODO.
\end{proof}
Now, recall the exact sequence \eqref{diag:sheaf_ex}. It looks awfully like
a beginning of an acyclic resolution if we take $U_{i}$ to be an affine
open cover of $X$. I should first check that intersections of affine
sets are affine. Then I will try to find sheaves whose global sections
agree with the objects in the exact sequence \eqref{diag:shea_ex}.
I hope to then extend the exact sequence in some natural way and check
that the resulting sheaves are in fact acyclic.

Let us begin by first introducing the following notation. Suppose $U_{i}$
is an affine open cover of some variety $X$. Then, denote
\[
  U_{i_{0},\ldots, i_{k}}=U_{i_{0}}\cap U_{i_{1}}\cap\cdots\cap U_{i_{k}}.
\]
Now I will check that intersections of affine sets are affine
\begin{prop}
  Given an affine cover $U_{i}$ of $X$, the intersections
  $U_{i_{0},\ldots,i_{k}}$ are affine and open.
\end{prop}
\begin{proof}
  TODO.
\end{proof}
Next I want to find a sheaf on $X$ whose global sections equal
$\mathcal{F}(U_{i_{0},\ldots,i_{k}})$. I simply take the sheaf
$i_{\ast}\mathcal{F}\vert_{U_{i_{0},\ldots,i_{k}}}$ defined by
\[
  i_{\ast}\mathcal{F}\vert_{U_{i_{0},\ldots,i_{k}}}(V)=\mathcal{F}
  (V\cap U_{i_{0},\ldots,i_{k}}).
\]
% TODO: Explain how we get the maps and continue the exact sequence
%       naturally.
We get a resolution
\[\begin{tikzcd}
    0\rar & \mathcal{F}\rar
    & \displaystyle\prod_{i_{0}}i_{\ast}\mathcal{F}\vert_{U_{i_{0}}}\rar
    & \displaystyle\prod_{i_{0},i_{1}}i_{\ast}\mathcal{F}\vert_{U_{i_{0},i_{1}}}\rar
    & \displaystyle\prod_{i_{0},i_{1},i_{2}}i_{\ast}\mathcal{F}\vert_{U_{i_{0},i_{1},i_{2}}}\rar
    & \cdots
  \end{tikzcd},\]
and I will finally check that the resolution is acyclic.
\begin{prop}
  The sheaves $\displaystyle\prod_{i_{0},\ldots,i_{k}}i_{\ast}\mathcal{F}
  \vert_{U_{i_{0},\ldots,i_{k}}}$ are $\Gamma$-acyclic.
\end{prop}
\begin{proof}
  TODO.
\end{proof}
Thus, we finally arrive at the definition of \v Cech cohomology.
\begin{defin}\label{def:cech}
  The \v Cech cohomology of a sheaf $\mathcal{F}$ of
  $\mathscr{O}_{X}$-modules is the cohomology of the cochain complex
  \[\begin{tikzcd}
      \mathcal{F}(U_{i_{0}})\rar{d^{0}} & \mathcal{F}(U_{i_{0},i_{1}})\rar{d^{1}}
      & \mathcal{F}(U_{i_{0},i_{1},i_{2}})\rar{d^{2}} & \cdots
    \end{tikzcd},\]
  where the differentials are defined as the product of the maps
  \[d_{i_{0},\ldots,i_{k+1}}^{k}:\displaystyle\prod_{j_{0},\ldots,j_{k}}
  \mathcal{F}(U_{j_{0},\ldots,j_{k}})\to \mathcal{F}(U_{i_{0},\ldots,i_{k+1}})\]
  given by
  \[
    d_{i_{0},\ldots,i_{k+1}}^{k}(\alpha)
    =\sum_{j=0}^{k+1}(-1)^{j}\alpha_{i_{0},\ldots,\hat{i}_{j},\ldots,i_{k+1}}
    \vert_{U_{i_{0},\ldots,i_{k+1}}}.
  \]
\end{defin}

\subsection{Results in cohomology}

\begin{prop}\label{prop:const_sheaf}
  If $X$ is an irreducible variety, and $A$ is an abelian group,
  then for the constant sheaf $\underline{A}$,
  \begin{enumerate}[(a)]
    \item $H^{0}(X,\underline{A}) = A$,
    \item $H^{1}(X,\underline{A})=0$.
  \end{enumerate}
\end{prop}
