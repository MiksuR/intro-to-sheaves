\section{Sheaf cohomology}
Studying sheaves using homological algebra turns out to be surprisingly
useful in many situations. For example, knowing that there is a
short exact sequence (SES)
\[
  \begin{tikzcd}
    0 \rar & \mathscr{F} \rar & \mathscr{G} \rar &
    \mathscr{H} \rar & 0
  \end{tikzcd}
\]
lets us relate the three sheaves together. By Thm.~\ref{thm:ses_equivalence}
this information is inherently \textbf{local} since this sequence is exact
if and only if the corresponding sequences on stalks are exact.
Then the question is: Can we get \textbf{global} information from
such exact sequences? We would hope that just as the sequence is exact
on stalks, it would also be exact on global sections:
\[
\begin{tikzcd}
  0 \rar & \Gamma(\mathscr{F}) \rar & \Gamma(\mathscr{G})
  \rar & \Gamma(\mathscr{H}) \rar & 0.
\end{tikzcd}
\]
Unfortunately, this is not the case. For example, let
$X=\mathbb{P}^{1}_{\mathbb{C}}$ and consider the sheaf morphism
$\mathscr{O}_{X}\to\mathbb{C}_{P_{0}}\oplus\mathbb{C}_{P_{1}}$,
which evaluates a section of $\mathscr{O}_{X}$ at some points $P_{0},P_{1}
\in X$. Then, the morphism is clearly surjective on the stalks. But it is
not surjective on global sections, since the global sections of
$\mathscr{O}_{X}$ are the constant functions. In other words, the exact
sequence
\[\begin{tikzcd}
    \mathscr{O}_{X}\rar & \mathbb{C}_{P_{0}}\oplus\mathbb{C}_{P_{1}}\rar & 0
  \end{tikzcd}\]
does not yield an exact sequence on global sections. However, we have the
following.
\begin{prop}
  If the sequence
  \[
  \begin{tikzcd}
    0\rar & \mathscr{F}\rar{\alpha} & \mathscr{G}\rar{\beta}
    & \mathscr{H}\rar & 0
  \end{tikzcd}
  \]
  is exact, then the sequence
  \[
  \begin{tikzcd}
    0\rar & \Gamma(\mathscr{F})\rar{\Gamma\alpha}
    & \Gamma(\mathscr{G})\rar{\Gamma\beta}
    & \Gamma(\mathscr{H})
  \end{tikzcd}
  \]
  is also exact.
\end{prop}
\begin{proof}\hfill
  \begin{description}[style=nextline]
    \item[Exactness at $\Gamma(\mathscr{F})$]
          Suppose $s\in\ker(\Gamma\alpha)$, which is to say that $s$ is in
          the kernel of the component $\alpha_{X}$ of the morphism $\alpha$.
          Since $\ker(\alpha)=\im(0\to\mathscr{F})$, we have
          $s\in\im({0\to\mathscr{F}})(X)$. But $\im(0\to\mathscr{F})$ is
          clearly the zero sheaf so that $s=0$. Also,
          $\im(0\to\Gamma(\mathscr{F}))$ is clearly the zero module.
          Therefore, $\ker(\Gamma\alpha)=0=\im({0\to\Gamma(\mathscr{F})})$.
    \item[Exactness at $\Gamma(\mathscr{G})$]
          Since $\alpha$ is injective, $\im(\alpha)$ can be identified with
          the sheaf $\mathscr{F}$. Therefore $\ker(\beta)=\mathscr{F}$ and so
          $\Gamma\left(\mathscr{F}\right)=\ker(\beta)(X)=\ker(\beta_{X})
          =\ker(\Gamma\beta)$. Since the second sequence is exact at
          $\Gamma\left(\mathscr{F}\right)$, the image $\im(\Gamma\alpha)$
          can also be identified with $\Gamma\left(\mathscr{F}\right)$.
          Therefore, $\im(\Gamma\alpha)=\Gamma(\mathscr{F})
          =\ker(\Gamma\beta)$.
  \end{description}

\end{proof}
\begin{cat}
  We say that the global sections functor $\Gamma: \text{Sh}(X)\to
  \mathscr{O}_{X}(X)$\textbf{-Mod} is left-exact but not right-exact.
\end{cat}

Although exact sequences are not completely preserved under taking
global sections, we won't give up! There might still be \textbf{a way of
measuring how much exactness fails}. We could measure the
\emph{obstruction} to exactness by continuing the sequence to the right
so that the following sequence is exact.
\[
\begin{tikzcd}
  0 \rar & \Gamma(\mathscr{F}) \rar & \Gamma(\mathscr{G})
  \rar\dar[phantom, ""{coordinate, name=Z}] & \Gamma(\mathscr{H})
  \arrow[rounded corners, to path={ -- ([xshift=2ex]\tikztostart.east)
    |- (Z) -| ([xshift=-2ex]\tikztotarget.west) -- (\tikztotarget)},
  overlay]{dll} & \\
    & H^{1}(\mathscr{F}) \rar & H^{1}(\mathscr{G})
  \rar & H^{1}(\mathscr{H}) \rar & \cdots \\
  \cdots \rar& H^{i}(\mathscr{F}) \rar & H^{i}(\mathscr{G})
  \rar & H^{i}(\mathscr{H}) \rar & \cdots.
\end{tikzcd}
\]
This problem of extending incomplete short exact sequences appears
elsewhere in homological algebra and is generally solved by constructing
so-called \emph{derived functors}. These vector spaces $H^{i}(-)$ given by
derived functors are then called the sheaf cohomology groups. In practise,
they are difficult to compute, and thus I will define the \emph{\v Cech
  cohomology} which is a tool for computing sheaf cohomology.
In the next subsections I will introduce derived functors and give a complete
explanation of how we arrive at \v Cech cohomology. The contents of these
subsections will be more technical than the rest of the paper, and it is
probably a good idea to skip straight to Definition~\ref{def:cech},
which can be taken as \textbf{the} definition of sheaf cohomology. The
primary source I use is \cite{vakil}.

\subsection{Derived functors*}
Let us consider the general problem of extending a left-exact
functor $F:\mathcal{A}\to\mathcal{B}$ between abelian categories
(which one may think of as categories of modules) to the right.
First, I will clarify what exactly I mean by a left-exact functor.
\begin{defin}
  Consider a functor $F:\mathcal{A}\to\mathcal{B}$ between abelian
  categories. The functor is said to be left-exact if
  \begin{enumerate}
    \item it is additive: if $A$, $B$ are objects of $\mathcal{A}$
          and $f, g\in\text{Hom}(A, B)$, then $F(f+g)=F(f)+F(g)$ and
    \item given a SES
          \[\begin{tikzcd}
              0\rar & A\rar & B\rar & C\rar & 0
            \end{tikzcd}\]
          of objects of $\mathcal{A}$, the sequence
          \[\begin{tikzcd}
              0\rar & F(A)\rar & F(B)\rar & F(C)
            \end{tikzcd}\]
          is exact.
  \end{enumerate}
\end{defin}
Now, for objects $A,B,C$ of $\mathcal{A}$ fitting into a SES
\[\begin{tikzcd}
    0\rar & A\rar & B\rar & C\rar & 0
  \end{tikzcd},\]
I want to find functors $R^{i}F:\mathcal{A}\to\mathcal{B}$ and
connecting morphisms so that the sequence
\[
\begin{tikzcd}
  0 \rar & F(A) \rar & F(B)
  \rar\dar[phantom, ""{coordinate, name=Z}] & F(C)
  \arrow[rounded corners, to path={ -- ([xshift=2ex]\tikztostart.east)
    |- (Z) -| ([xshift=-2ex]\tikztotarget.west) -- (\tikztotarget)},
  overlay]{dll} & \\
  & R^{1}F(A)\rar & R^{1}F(B)\rar
  & R^{1}F(C)\rar & \cdots \\
  \cdots\rar & R^{i}F(A)\rar & R^{i}F(B)\rar
  & R^{i}F(C)\rar & \cdots.
\end{tikzcd}
\]
is exact. The functors $R^{i}F$ will be called the \emph{right
  derived functors} of $F$. The following lemma from homological
algebra gives a hint of what approach one should take to find such functors.
\begin{lemm}[Zig-zag lemma] % TODO: Cite this?
  Suppose $A^{\bullet},B^{\bullet},C^{\bullet}$ are cochain complexes in some
  abelian category. If there is a SES
  \[\begin{tikzcd}
      0\rar & A^{\bullet}\rar & B^{\bullet}\rar & C^{\bullet}\rar & 0
    \end{tikzcd},\]
  then there are maps between the cohomology groups of these complexes
  such that the sequence
  \[\begin{tikzcd}
    & H^{0}(A^{\bullet}) \rar & H^{0}(B^{\bullet})
    \rar\dar[phantom, ""{coordinate, name=Z}] & H^{0}(C^{\bullet})
    \arrow[rounded corners, to path={ -- ([xshift=2ex]\tikztostart.east)
      |- (Z) -| ([xshift=-2ex]\tikztotarget.west) -- (\tikztotarget)},
    overlay]{dll} & \\
    & H^{1}(A^{\bullet}) \rar & H^{1}(B^{\bullet})\rar
    & H^{1}(C^{\bullet}) \rar & \cdots \\
    \cdots \rar& H^{i}(A^{\bullet}) \rar & H^{i}(B^{\bullet})\rar
    & H^{i}(C^{\bullet}) \rar & \cdots
    \end{tikzcd}\]
  is exact.
\end{lemm}
This lemma can be shown using a typical diagram chasing argument: The maps
$H^{i}(A^{\bullet})\to H^{i}(B^{\bullet})$ and
$H^{i}(B^{\bullet})\to H^{i}(C^{\bullet})$ are given by functoriality, and the
connecting morphisms $H^{i}(C^{\bullet})\to H^{i+1}(A^{\bullet})$ are given by
the snake lemma. Working out the details of this diagram chasing argument is
a good exercise for the reader, but I will instead prove the statement using
specrtal sequences. Explaining the theory of spectral sequences is beyond
the scope of this paper and I will refer the reader to the section 1.7 of
\cite{vakil}.
\begin{proof}
  Define the zeroth page of a spectral sequence to be the following
  double complex given by the SES of complexes.
  \[\begin{tikzcd}
      & \vdots & \vdots & \vdots & \\
      0\rar & A^{2}\rar\uar & B^{2}\rar\uar & C^{2}\rar\uar & 0 \\
      0\rar & A^{1}\rar\uar & B^{1}\rar\uar & C^{1}\rar\uar & 0 \\
      0\rar & A^{0}\rar\uar & B^{0}\rar\uar & C^{0}\rar\uar & 0 \\
      & 0\uar & 0\uar & 0\uar &
    \end{tikzcd}\]
  Since the rows are exact the first page is zero when we use the rightward
  orientation. Now, let us compute the first page using upward orientation.
  We get the following.
  \[\begin{tikzcd}
      & \vdots & \vdots & \vdots & \\
      0\rar & H^{2}(A^{\bullet})\rar{\alpha_{2}}
      & H^{2}(B^{\bullet})\rar{\beta_{2}} & H^{2}(C^{\bullet})\rar & 0 \\
      0\rar & H^{1}(A^{\bullet})\rar{\alpha_{1}}
      & H^{1}(B^{\bullet})\rar{\beta_{1}} & H^{1}(C^{\bullet})\rar & 0 \\
      0\rar & H^{0}(A^{\bullet})\rar{\alpha_{0}}
      & H^{0}(B^{\bullet})\rar{\beta_{0}} & H^{0}(C^{\bullet})\rar & 0 \\
    \end{tikzcd}\]
  Finally, in the second page we have

  \begin{center}
  \begin{tikzpicture}[commutative diagrams/every diagram, x=2.2cm, y=1.7cm]
    \clip (0.8,0.5) rectangle (5.2,3.5);
    \node (A0) at (2,1) {$\ker(\alpha_{0})$};
    \node (B0) at (3,1) {$\frac{\ker(\beta_{0})}{\im(\alpha_{0})}$};
    \node (C0) at (4,1) {$\coker(\beta_{0})$};
    \node (A1) at (2,2) {$\ker(\alpha_{1})$};
    \node (B1) at (3,2) {$\frac{\ker(\beta_{1})}{\im(\alpha_{1})}$};
    \node (C1) at (4,2) {$\coker(\beta_{1})$};
    \node (A2) at (2,3) {$\ker(\alpha_{2})$};
    \node (B2) at (3,3) {$\frac{\ker(\beta_{2})}{\im(\alpha_{2})}$};
    \node (C2) at (4,3) {$\coker(\beta_{2})$};
    \node (lu) at (1,3) {$0$};
    \node (lm) at (1,2) {$0$};
    \node (rd) at (5,1) {$0$};
    \node (rm) at (5,2) {$0$};

    \path[commutative diagrams/.cd, every arrow, every label]
      (0,4) edge (A2)
      (0,3) edge (A1)
      (0,2) edge (A0)
      (lu) edge (B1)
      (lm) edge (B0)
      (1,4) edge (B2)
      (2,4) edge (C2)
      (A2) edge (C1)
      (A1) edge (C0)
      (B2) edge (rm)
      (B1) edge (rd)
      (C2) edge (6,2)
      (C1) edge (6,1)
      (C0) edge (6,0)
      (A0) edge (4,0)
      (B0) edge (5,0);
  \end{tikzpicture}
  \end{center}
  One can see that the spectral sequence will converge on the third
  page. Since the sequence converges to zero, the sequences
  \[\begin{tikzcd}
      0\rar & \ker(\alpha_{i+1})\rar & \coker(\beta_{i})\rar & 0,
    \end{tikzcd}\]
  given by the differentials on the second page must be exact.
  These isomorphisms induce maps
  \[\delta_{i}:H^{i}(C^{\bullet})\to H^{i+1}(A^{\bullet}).\]
  The convergence of the spectral sequence also implies that
  $\ker(\beta_{i})/\im(\alpha_{i})=0$. Putting these results together,
  we see that the sequence
  \[\begin{tikzcd}
    & H^{0}(A^{\bullet})\rar{\alpha_{0}} & H^{0}(B^{\bullet})
    \rar{\beta_{0}}\dar[phantom, ""{coordinate, name=Z}] & H^{0}(C^{\bullet})
    \arrow[rounded corners, to path={[pos=0] --
      ([xshift=2ex]\tikztostart.east) |- (Z) -|
      ([xshift=-2ex]\tikztotarget.west)\tikztonodes -- (\tikztotarget)},
    "\delta_{0}"']{dll} & \\
    & H^{1}(A^{\bullet})\rar{\alpha_{1}} & H^{1}(B^{\bullet})\rar{\beta_{1}}
    & H^{1}(C^{\bullet})\rar{\delta_{1}} & \cdots \\
    \cdots \rar{\delta_{i-1}}& H^{i}(A^{\bullet})\rar{\alpha_{i}}
    & H^{i}(B^{\bullet})\rar{\beta_{i}} & H^{i}(C^{\bullet})\rar{\delta_{i}}
    & \cdots
    \end{tikzcd}\]
  is exact.
\end{proof}

Therefore, in order to extend the sequence
\[\begin{tikzcd}
    0\rar & F(A)\rar & F(B)\rar & F(C)
  \end{tikzcd},\]
we wish to find cocomplexes $A^{\bullet}, B^{\bullet}, C^{\bullet}$
associated to $A, B, C$ such that
\begin{enumerate}
  \item The cochain complexes $A^{\bullet}, B^{\bullet}, C^{\bullet}$ fit into a
        SES
        \[\begin{tikzcd}
            0\rar & A^{\bullet}\rar & B^{\bullet}\rar & C^{\bullet}\rar & 0
          \end{tikzcd}\]
  \item The zeroth cohomology coincides with $F$:
        \[H^{0}(A^{\bullet})=F(A),
        \quad H^{0}(B^{\bullet})=F(B),
        \quad H^{0}(C^{\bullet})=F(C).\]
\end{enumerate}

One way of associating a cochain complex to an object $A$ is to take its
\emph{resolution}. In other words, by finding objects $A^{i}$ and
morphisms such that the sequence
\[\begin{tikzcd}
    0\rar & A\rar{\alpha} & A^{0}\rar{\alpha_{0}} & A^{1}\rar{\alpha_{1}}
    & A^{2}\rar{\alpha_{2}} & \cdots
  \end{tikzcd}\]
is exact. Let us concentrate on the first few terms of this sequence.
\[\begin{tikzcd}
    0\rar & A\rar{\alpha} & A^{0}\rar{\alpha_{0}} & A^{1}
  \end{tikzcd}.\]
Since $F$ is left-exact, we get an another exact sequence:
\[\begin{tikzcd}
    0\rar & F(A)\rar{\alpha^{\ast}}
    & F(A^{0})\rar{\alpha_{0}^{\ast}} & F(A^{1})
  \end{tikzcd}.\]
Now, exactness implies that $\ker(\alpha_{0}^{\ast})=\im(\alpha^{\ast})
=F(A)$. Therefore, if I were to replace $F(A)$ by $0$,
then the cohomology at $F(A^{0})$ would be $F(A)$!
Thus, considering the cochain complex
\[\begin{tikzcd}
    0\rar & F(A^{0})\rar{\alpha_{0}^{\ast}}
    & F(A^{1})\rar{\alpha_{1}^{\ast}}
    & F(A^{2})\rar{\alpha_{2}^{\ast}} & \cdots
  \end{tikzcd},\]
we see that taking the cohomology of this complex will give
\[H^{0}(F(A^{\bullet}))=F(A).\]
Hence, if I construct the cochain complexes $A^{\bullet},B^{\bullet},
C^{\bullet}$ from resolutions of $A,B,C$ as above, then the resulting
cohomolgy will satisfy the second requirement. Next, I want to find the right
type of resolution so that the cochain complexes satisfy the first
requirement above.

What we have currently is the following picture.
\[\begin{tikzcd}
    & \vdots & \vdots & \vdots & \\
    & A^{1}\uar{\alpha_{1}} & B^{1}\uar{\beta_{1}} & C^{1}\uar{\gamma_{1}} & \\
    & A^{0}\uar{\alpha_{0}} & B^{0}\uar{\beta_{0}} & C^{0}\uar{\gamma_{0}} & \\
    0\rar& A\rar{f}\uar{\alpha} & B\rar{g}\uar{\beta} & C\rar\uar{\gamma}&0\\
    & 0\uar & 0\uar & 0\uar &
  \end{tikzcd}\]
If I use \emph{injective resolutions}, the diagram can be filled in with
approprate morphisms so that it gives a SES of complexes.
\begin{defin}
  An object $I$ of an abelian category $\mathcal{A}$ is injective,
  if for every injection $f:A\hookrightarrow B$ and every morphism
  $g:A\to I$, there is a morphism $B\to I$ such that the following
  diagram commutes.
  \[\begin{tikzcd}[row sep=large]
      I & \\ A\uar{g}\arrow[r, hook, "f"'] & B\ular[dashed]
    \end{tikzcd}\]
\end{defin}
Then, an injective resolution of an object $A$ is a long exact sequence
\[\begin{tikzcd}
    0\rar & A\rar & I^{0}\rar & I^{1}\rar & \cdots
  \end{tikzcd}\]
where the $I^{i}$ are injective. Note that it is not obvious that an object
of an abelian category should have an injective resolution in the first
place. Thus, we assume that the category $\mathcal{A}$ has
\emph{enough injectives} meaning that for every object $A$ of $\mathcal{A}$,
there is an injection $A\hookrightarrow I$ into some injective object $I$.
Then, for every object $A$ we can construct an injective resolution
inductively. The first object $I^{0}$ is given directly by the assumption.
Then suppose we have constructed an exact sequence
\[\begin{tikzcd}
    0\rar & A\rar & I^{0}\rar{\iota_{0}} & \cdots\rar
    & I^{n-1}\rar{\iota_{n-1}} & I^{n}
  \end{tikzcd}.\]
Let us take $I^{n+1}$ to be an injective object such that there is
an injection ${\coker(\iota_{n-1})\hookrightarrow I^{n+1}}$. Then we have
\[\begin{tikzcd}
    0\rar & A\rar & I^{0}\rar{\iota_{0}} & \cdots\rar
    & I^{n-1}\rar{\iota_{n-1}} & I^{n}\rar[two heads]
    & \coker(\iota_{n-1})\rar[hook] & I^{n+1}
  \end{tikzcd}.\]
When the injection is composed with the projection, we get the exact sequence
\[\begin{tikzcd}
    0\rar & A\rar & I^{0}\rar{\iota_{0}} & \cdots\rar
    & I^{n-1}\rar{\iota_{n-1}} & I^{n}\rar{\iota_{n}} & I^{n+1}
  \end{tikzcd}.\]
Now I will quickly prove a useful lemma about injective objects:
\begin{lemm}\label{lemm:product_injectives}
  A product of injective objects is injective.
\end{lemm}
\begin{proof}
  Suppose $(J_{i})_{i\in I}$ is a collection of injective objects in an
  abelian category, and denote
  \[
    J=\bigoplus_{i\in I} J_{i}.
  \]
  Then, suppose we have a morphism $g: X\to J$ and an injection
  $f:X\hookrightarrow Y$. Recall that $J$ is injective if we can lift $g$
  along $f$ so that the following diagram commutes.
  \[\begin{tikzcd}[row sep=large, column sep=large]
      J & \\ X\uar{g}\arrow[r, hook, "f"'] & Y\ular[dashed]
    \end{tikzcd}\]
  We can compose $g$ with the projection morphisms
  $\pi_{J_{i}}:J\to J_{i}$, and then we can lift the compositions along $f$
  since the $J_{i}$ are injective by assumption:
  \[\begin{tikzcd}[row sep=large, column sep=large]
        J_{i} & \\ X\uar{\pi_{J_{i}}\circ g}\arrow[r, hook, "f"']
        & Y\arrow[ul, dashed, "\ell_{J_{i}}"']
    \end{tikzcd}\]
  Now, by the universal property of the product $J$, there is a morphism
  $\ell: Y\to J$, such that the following diagrams commute for all $i\in I$.
  \[\begin{tikzcd}[row sep=large, column sep=large]
      J\arrow[d, "\pi_{J_{i}}"'] & Y\arrow[l, dashed, "\ell"']
      \dlar{\ell_{J_{i}}} & X\arrow[l, hook, "f"']
      \arrow[dll, bend left=20, "\pi_{J_{i}\circ g}"] \\ J_{i} &&
    \end{tikzcd}\]
  By the universal property of the product $J$, we also have that $g$ is the
  unique morphism making the triangles
  \[\begin{tikzcd}[row sep = large, column sep=large]
      J\arrow[d, "\pi_{J_{i}}"'] & X\arrow[l, "g"']\dlar{\pi_{J_{i}\circ g}} \\
      J_{i} &
    \end{tikzcd}\]
  commute for all $i\in I$. Since these triangles are the same as the ones
  we get by composing $\ell$ with $f$, we must have $\ell\circ f=g$
  and therefore, $\ell$ is a lift of $g$ along $f$.
\end{proof}
Now, the definition of injective objects gives a way of constructing a SES
of injective resolutions given a SES of objects in an abelian category.
\begin{lemm}\label{lemm:ses_of_resolutions}
  If $A, B, C$ are objects of an abelian category $\mathcal{A}$ with enough
  injectives fitting into a SES
  \[\begin{tikzcd}
      0\rar & A\rar{f} & B\rar{g} & C\rar & 0 \\
    \end{tikzcd}\]
  and $A$ and $C$ have injective resolutions $A^{\bullet}$ and $C^{\bullet}$,
  then there is an injective resolution $B^{\bullet}$ of $B$ and maps
  $f^{i}:A^{i}\to B^{i}$ and $g^{i}:B^{i}\to C^{i}$ such that the following
  commutative diagram has exact rows.
\[\begin{tikzcd}
    & \vdots & \vdots & \vdots & \\
    0\rar & A^{1}\rar{f^{1}}\uar{\alpha_{1}} & B^{1}\rar{g^{1}}\uar{\beta_{1}}
    & C^{1}\rar\uar{\gamma_{1}} & 0 \\
    0\rar & A^{0}\rar{f^{0}}\uar{\alpha_{0}} & B^{0}\rar{g^{0}}\uar{\beta_{0}}
    & C^{0}\rar\uar{\gamma_{0}} & 0 \\
    0\rar& A\rar{f}\uar{\alpha} & B\rar{g}\uar{\beta} & C\rar\uar{\gamma}&0\\
    & 0\uar & 0\uar & 0\uar &
  \end{tikzcd}\]
\end{lemm}
\begin{proof}
  One can take $B^{i}=A^{i}\oplus C^{i}$, and then the rows
  \[\begin{tikzcd}[row sep=0]
      0\rar & A^{i}\rar & A^{i}\oplus C^{i}\rar & C^{i}\rar & 0 \\
      & a\rar[mapsto] & (a, 0) && \\
      && (a, c)\rar[mapsto] & c &
    \end{tikzcd}\]
  are clearly exact. Moreover, the $B^{i}$ are injective by
  Lemma~\ref{lemm:product_injectives}, so I only need to find maps
  $\beta:B\to B^{0}$ and $\beta^{i}:B^{i}\to B^{i+1}$ making the diagram
  commute so that the $B^{i}$ form an injective resolution of $B$.

  The morphism $\beta:B\to B^{0}$ making the below diagram commute is
  constructed as follows.
  \[\begin{tikzcd}
      0\rar & A^{0}\rar{f^{0}} & A^{0}\oplus C^{0}\rar{g^{0}} & C^{0}\rar & 0 \\
      0\rar & A\rar{f}\uar{\alpha} & B\rar{g}\uar{\beta}
      & C\rar\uar{\gamma} & 0 \\ & 0\uar & 0\uar & 0\uar &
    \end{tikzcd}\]
  commutes. Since $A^{0}$ is injective, we can lift $\alpha$ along the
  injection $f$:
  \[\begin{tikzcd}
      A^{0} & \\ A\arrow[r, "f"']\uar{\alpha}
      & B\arrow[ul, dashed, "\alpha^{\prime}"']
    \end{tikzcd}\]
  Then, we get the diagram
  \[\begin{tikzcd}[row sep=large, column sep=large]
      A^{0} & A^{0}\oplus C^{0}\arrow[l, "\pi_{A^{0}}"']\rar{\pi_{C^{0}}}
      & C^{0} \\ & B\ular{\alpha^{\prime}}\uar[dashed]{\beta}
      \arrow[ur, "\gamma\circ g"']
    \end{tikzcd},\]
  where $\beta$ is given by the universal property of the product.
  The morphisms $\beta_{i}$ are constructed in exactly the same way.

  Finally, I need to check the exactness of the sequence
  \[\begin{tikzcd}
      0\rar & B\rar{\beta} & B^{0}\rar{\beta_{0}} & B^{1}\rar{\beta_{1}} & \cdots
    \end{tikzcd},\]
  but this follows by nearly trivial spectral sequence argument.
  Define a specral sequence where the 0th page is the double complex we have
  constructed. Computing the 1st page using rightward orientation yields 0,
  since all rows are exact. Thus, the spectral sequence converges to 0.
  One can see that the spectral sequence must converge on the 1st page
  also when using the upward orientation. Therefore, the column corresponding
  to the injective resolution of $B$ must be exact.
\end{proof}
Now I only need to apply the functor $F$ on this double complex
and remove the bottom row. The only worry is that the rows don't stay
exact. Thus, I will need to prove one more small result.
\begin{lemm}
  Suppose $F:\mathcal{A}\to\mathcal{B}$ is a left-exact functor between
  abelian categories. If $I$ and $J$ are injective objects of $\mathcal{A}$,
  then $F$ is exact on the SES
  \[\begin{tikzcd}
      0\rar & I\rar & I\oplus J\rar & J\rar & 0
    \end{tikzcd}.\]
\end{lemm}
\begin{proof}
  In abelian categories, the object $I\oplus J$ is both a product and
  a coproduct. It follows that additive functors preserve these products.
  The result follows directly from this remark, because applying the functor
  $F$ on the SES yields
  \[\begin{tikzcd}
      0\rar & F(I)\rar & F(I)\oplus F(J)\rar & F(J)\rar & 0
    \end{tikzcd},\]
  which is clearly exact.
\end{proof}

In summary, given a left-exact functor $F:\mathcal{A}\to
\mathcal{B}$ and an object $A$ of $\mathcal{A}$, one constructs the $i$th
right derived functor of $F$ at $A$ in the following way:
\begin{enumerate}
  \item Find an injective resolution $0\to A\to I^{\bullet}$ of $A$
  \item Apply $F$ on the cochain complex $0\to I^{\bullet}$
  \item Take the $i$th cohomology of the resulting cochain complex
        \[\begin{tikzcd}
            0\rar & F(I^{0})\rar & F(I^{1})\rar
            & F(I^{2})\rar & \cdots
          \end{tikzcd}\]
\end{enumerate}
We denote $R^{i}F(A)=H^{i}(F(I^{\bullet}))$ for the value of the
derived functor.
\begin{lwarn}
  The alert reader might have noticed that this definition depends a-priori
  on the injective resolution we choose. However, one can show that this is
  not the case, see \cite{vakil}.
\end{lwarn}
\begin{bcat}
  One could ask: ``How do we know that derived functors give the `right'
  way of extending the left-exact functor to the right?'' This can be
  formalised by considering so-called (cohomological) $\delta$-functors,
  which consist of pairs $(T^{i},\delta^{i})$, where
  \begin{enumerate}
    \item The $T^{i}:\mathcal{A}\to\mathcal{B}$ are additive functors between
          abelian categories with $T^{i}=0$ for $i<0$, and
    \item $\delta^{i}:T^{i}(C)\to T^{i+1}(A)$ are morphisms in $\mathcal{B}$,
  \end{enumerate}
  such that for every SES
  \[\begin{tikzcd}
      0\rar & A\rar & B\rar & C\rar & 0
    \end{tikzcd}\]
  in $\mathcal{A}$, we have a long exact sequence
  \[\begin{tikzcd}
      0\rar & T^{0}(A)\rar & T^{0}(B)\rar\dar[phantom, ""{coordinate, name=Z}]
      & T^{0}(C)\arrow[rounded corners, to path={[pos=0] --
      ([xshift=2ex]\tikztostart.east) |- (Z) -|
      ([xshift=-2ex]\tikztotarget.west)\tikztonodes -- (\tikztotarget)},
    "\delta_{0}"']{dll} & \\
    & T^{1}(A)\rar & T^{1}(B)\rar & T^{1}(C)\rar{\delta_{1}} & \cdots \\
    \cdots \rar{\delta_{i-1}}& T^{i}(A)\rar & T^{i}(B)\rar
    & T^{i}(C)\rar{\delta_{i}} & \cdots
    \end{tikzcd}\]
  In addition, we require functoriality of this construction: If
  \[\begin{tikzcd}
      0\rar & A\rar\dar & B\rar\dar & C\rar\dar & 0 \\
      0\rar & A^{\prime}\rar & B^{\prime}\rar & C^{\prime}\rar & 0
    \end{tikzcd}\]
  is a morphism of short exact sequences in $\mathcal{A}$, then the
  squares
  \[\begin{tikzcd}
      T^{i}(C)\rar{\delta^{i}}\dar & T^{i+1}(A)\dar \\
      T^{i}(C^{\prime})\arrow["\delta^{i}"']{r} & T^{i+1}(A^{\prime})
    \end{tikzcd}\]
  commute.

  One can then define morphisms of $\delta$-functors so that they form
  a category. Then, one can define the concept of a \emph{universal}
  $\delta$\emph{-functor} in this category: A $\delta$-functor
  $(T^{i},\delta^{i})$ is universal if for every other $\delta$-functor
  $(S^{i},\gamma^{i})$ with a natural transformation
  $\alpha: T^{0}\Rightarrow S^{0}$, there is a unique morphism of
  $\delta$-functors $(T^{i},\delta^{i})\to (S^{i},\gamma^{i})$ extending
  $\alpha$. Then, one can prove that derived functors define a universal
  $\delta$-functor. I will not spell out the details here, but further
  information on $\delta$-functors can be found in \cite{vakil}.
\end{bcat}
Now, consider a sheaf $\mathscr{F}$ of $\mathscr{O}_{X}$-modules.
Its \emph{sheaf cohomology} is then defined as the right derived functor of
the global sections functor:
\[
  H^{i}(X, \mathscr{F}) := R^{i}\Gamma\left(\mathscr{F}\right).
\]
This definition relies on the assumption that the category of sheaves of
$\mathscr{O}_{X}$-modules has enough injectives.
% TODO: Assume enough injectives in R-Mod and derive the theorem from this.
\begin{thm}\label{thm:enough_injectives}
  The category $\mathscr{O}_{X}$\textup{\textbf{-Mod}} has enough injectives.
\end{thm}
\begin{proof}
  The proof of this theorem is given as a series of exercises in
  \cite{vakil}.
\end{proof}

\subsection{Acyclic resolutions*}
Derived functors provide the framework for the cohomology theory of
sheaves, but working with injective resolutions is difficult in practise.
Thus, we need an alternative way of constructing the cohomology groups.
I will do this by replacing injective resolutions by \emph{acyclic
  resolutions}. If $F:\mathcal{A}\to\mathcal{B}$ is a left-exact
functor between abelian categories, then an object $A$ of $\mathcal{A}$ is
acyclic if $R^{i}F(A)=0$ for $i>0$. I will justify this in the
next lemma.
\begin{lemm}
  Let $F:\mathcal{A}\to\mathcal{B}$ be a left-exact functor
  between abelian categories, where $\mathcal{A}$ has enough injectives.
  Suppose $A$ is an object of $\mathcal{A}$ with acyclic resolution
  \[\begin{tikzcd}
      0\rar & A\rar & A^{0}\rar & A^{1}\rar & \cdots
    \end{tikzcd}.\]
  Then, computing the cohomology of the cochain complex
  \[\begin{tikzcd}
      0\rar & F(A^{0})\rar & F(A^{1})\rar & \cdots
    \end{tikzcd}\]
  agrees with the right derived functor of $F$ at $A$.
\end{lemm}
\begin{proof}
  I will prove the statement using a spectral sequence argument.
  Thus, I need to set up a double complex, which will be the 0th page
  of the sequence. First note that the long exact sequence
  \[\begin{tikzcd}
      0\rar & A\rar{\alpha} & A^{0}\rar{\alpha_{0}} & A^{1}\rar{\alpha_{1}}
      & \cdots
    \end{tikzcd}\]
  can be broken into short exact sequences
  \[\begin{tikzcd}
      0\rar & A\rar{\alpha} & \im\alpha\rar & 0
    \end{tikzcd}\]
  \begin{center}
    and
  \end{center}
  \[\begin{tikzcd}
      0\rar & \ker\alpha_{i}\rar & A^{i}\rar{\alpha_{i}} & \im\alpha_{i}\rar
      & 0
    \end{tikzcd}.\]
  Lemma~\ref{lemm:ses_of_resolutions} can now be used to construct short
  exact sequences of resolutions, which can be combined to form a double
  complex.
  \[\begin{tikzcd}
      & \vdots & \vdots & \vdots & \vdots & \\
      0\rar & I^{2}\uar\rar & I^{0, 2}\uar\rar & I^{1, 2}\uar\rar
      & I^{2, 2}\uar\rar & \cdots \\
      0\rar & I^{1}\uar\rar & I^{0, 1}\uar\rar & I^{1, 1}\uar\rar
      & I^{2, 1}\uar\rar & \cdots \\
      0\rar & I^{0}\uar\rar & I^{0, 0}\uar\rar & I^{1, 0}\uar\rar
      & I^{2, 0}\uar\rar & \cdots \\
      0\rar & A\uar\rar & A^{0}\uar\rar & A^{1}\uar\rar & A^{2}\uar\rar
      & \cdots \\
      & 0\uar & 0\uar & 0\uar & 0\uar &
    \end{tikzcd}\]
  Removing the bottom row and left-most column and applying the functor
  $F$ gives the following double complex.
  \[\begin{tikzcd}
      & \vdots & \vdots & \vdots & \\
      0\rar & F(I^{0,2})\uar\rar & F(I^{1, 2})\uar\rar
      & F(I^{2, 2})\uar\rar & \cdots \\
      0\rar & F(I^{0,1})\uar\rar & F(I^{1, 1})\uar\rar
      & F(I^{2, 1})\uar\rar & \cdots \\
      0\rar & F(I^{0,0})\uar\rar & F(I^{1, 0})\uar\rar
      & F(I^{2, 0})\uar\rar & \cdots \\
      & 0\uar & 0\uar & 0\uar &
    \end{tikzcd}\]
  Now, I take this double complex to be the 0th page of the spectral
  sequence. Note that computing the cohomology groups of the rows and columns
  is the same as computing the values of the right derived functors of
  $F$; The cohomology groups of the rows correspond to the values
  of the right derived functor at the objects $I^{i}$ and the cohomology
  groups of the columns correspond to the values at $A^{i}$. Note also that
  injective objects are acyclic, because an injective object $I$ has the
  trivial resolution $0\to I\to I\to 0$. Thus, when one computes the first
  page of the sequence starting with the rightward orientation, the only
  non-zero column will be the first column:
  \[\begin{tikzcd}
      \vdots \\ F(I^{1})\uar \\ F(I^{0})\uar \\ 0\uar
    \end{tikzcd}\]
  Then, the entries on the second page will be equal to the values of the
  right derived functors $R^{i}F$ at the object $A$.
  \begin{center}
  \begin{tikzpicture}[commutative diagrams/every diagram, x=2cm, y=1.2cm]
    \clip (0.8,0.5) rectangle (3.2,3.5);
    \node (R) at (2,0) {$R^{-1}F(A)$};
    \node (R0) at (2,1) {$R^{0}F(A)$};
    \node (R1) at (2,2) {$R^{1}F(A)$};
    \node (R2) at (2,3) {$R^{2}F(A)$};
    \node (R3) at (2,4) {$R^{3}F(A)$};
    \node (ld) at (1,1) {$0$};
    \node (lu) at (1,3) {$0$};
    \node (lm) at (1,2) {$0$};
    \node (rd) at (3,1) {$0$};
    \node (rm) at (3,2) {$0$};
    \node (ru) at (3,3) {$0$};

    \path[commutative diagrams/.cd, every arrow, every label]
    (R) edge (lm)
    (R0) edge (lu)
    (rd) edge (R2)
    (rm) edge (R3)
    (R1) edge (1,4)
    (3,0) edge (R1);
  \end{tikzpicture}
  \end{center}
  The sequence collapses at the 2nd step and one can see that the total
  cohomology of the complex corresponds to the values of the right
  derived functor at $A$. Similarly, computing the first page starting with
  the upward orientation gives only one non-zero row, since the $A^{i}$
  are acyclic:
  \[\begin{tikzcd}
      0\rar & F(A^{0})\rar & F(A^{1})\rar
      & F(A^{2})\rar & \cdots
    \end{tikzcd}\]
  Computing the second page gives the cohomology groups of this complex.
  Again, the sequence collapses and therefore the values of the right
  derived functor agree with the cohomology groups of the complex.
\end{proof}
Let me quickly prove a lemma about products of acyclic objects.
\begin{lemm}\label{lemm:acyclic_prod}
  Suppose $F:\mathcal{A}\to\mathcal{B}$ is a left-exact functor between
  abelian categories. If $A_{1}$ and $A_{2}$ are $F$-acyclic objects of
  $\mathcal{A}$, then the product $A_{1}\oplus A_{2}$ is $F$-acyclic.
\end{lemm}
\begin{proof}
  Suppose $A_{1}$ and $A_{2}$ have injective resolutions
  \[\begin{tikzcd}
    0\rar & A_1\rar & I_1^0\rar{d_1^0} & I_1^1\rar{d_1^1} &
    I_1^2\rar{d_1^2} & \cdots
  \end{tikzcd}\]
  \begin{center}
    and
  \end{center}
  \[\begin{tikzcd}
    0\rar & A_2\rar & I_2^0\rar{d_2^0} & I_2^1\rar{d_2^1}
    & I_2^2\rar{d_2^2} & \cdots
  \end{tikzcd}.\]

  Then,
  \[\begin{tikzcd}
      0\rar & A_{1}\oplus A_{2}\rar & I_{1}^{0}\oplus I_{2}^{0}
      \rar{d_{1}^{0}\oplus d_{2}^{0}} & I_{1}^{1}\oplus I_{2}^{1}
      \rar{d_{1}^{1}\oplus d_{2}^{1}} & \cdots
    \end{tikzcd}\]
  is an injective resolution of $A_{1}\oplus A_{2}$ by
  Lemma~\ref{lemm:product_injectives}. Applying the functor $F$ to the
  cochain complex $0\to I_{1}^{\bullet}\oplus I_{2}^{\bullet}$ yields
  \begin{equation}\label{comp:prod_resolution}
    \begin{tikzcd}[column sep=large]
      0\rar & F(I_{1}^{0})\oplus F(I_{2}^{0})
      \rar{F(d_{1}^{0})\oplus F(d_{2}^{0})} & F(I_{1}^{1})\oplus F(I_{2}^{1})
      \rar{F(d_{1}^{1})\oplus F(d_{2}^{1})} & \cdots
    \end{tikzcd},
  \end{equation}
  as additive functors preserve finite products. Since $A_{1}$ and $A_{2}$
  are $F$-acyclic, the cochain complexes $0\to F\left(I_{1}^{\bullet}\right)$
  and $0\to F\left(I_{2}^{\bullet}\right)$ are exact at
  $F\left(I_{1}^{i}\right)$ and $F\left(I_{2}^{i}\right)$ for $i>0$.
  Therefore, one can immediately see that the cochain complex
  \eqref{comp:prod_resolution} is exact at
  $F\left(I_{1}^{i}\right)\oplus F\left(I_{2}^{i}\right)$ for $i>0$.
\end{proof}

Next, I want to find an acyclic resolution for a given sheaf so that I can
calculate its sheaf cohomology. Recall Prop.~\ref{prop:qcoh_gsec_exact},
which states that if we have a SES of quasi-coherent sheaves on an affine
variety, then the global sections functor is exact on this SES. This leads us
to suspect that quasi-coherent sheaves on affine varieties are acyclic,
which is indeed the case. The proof of this statement relies on the theory
of flasque sheaves, which I have not covered.
\begin{lemm}[Thm. 3.5 in \cite{hartshorne}]\label{lemm:affine_cohom_vanishes}
  If $X$ is an affine variety and $\mathscr{F}$ is a quasi-coherent
  sheaf, then $H^{i}(X, \mathscr{F})=0$ for all $i>0$.
\end{lemm}
\begin{proof}
  If I denote $M=\Gamma(\mathscr{F})$, then $\mathscr{F}=\cm$. Now, the
  module $M$ has an injective resolution $0\to M\to I^{\bullet}$.
  This gives an exact sequence $0\to \cm\to\widetilde{I}^{\bullet}$.
  The sheaves $\widetilde{I}^{i}$ are acyclic by Prop. 3.4 and Prop. 2.5
  in \cite{hartshorne} (injective sheaves are flasque and flasque sheaves
  are acyclic). Therefore, the sequence $0\to \cm\to\widetilde{I}^{\bullet}$
  is an acyclic resolution of $\cm=\mathscr{F}$. Applying the global sections
  functor to this resolution returns the original sequence
  $0\to M\to I^{\bullet}$, which is exact. After removing the $M$ term, the
  cohomology groups at $I^{0}$ is $\Gamma(\mathscr{F})$, but the cohomolgy
  groups at $I^{i}$ are zero for $i>0$.
\end{proof}
Now, consider a quasi-coherent sheaf $\mathscr{F}$ on a general variety $X$
and recall the exact sequence \eqref{diag:sheaf_ex}. If we take $U=X$
and the $U_{i}$ form an affine open cover of $X$, then the sequence reads
\[\begin{tikzcd}
    0\rar & \mathscr{F}(X)\rar
    & \displaystyle\prod_{i_{0}}\mathscr{F}(U_{i_{0}})\rar
    & \displaystyle\prod_{i_{0},i_{1}}\mathscr{F}(U_{i_{0}}\cap U_{i_{1}})
  \end{tikzcd},\]
which looks awfully like the beginning of an acyclic resolution in light of
the above lemma. Thus, I would like to find sheaves with global sections
that match the ones in the exact sequence. But this is quite simple in fact;
I can take the sheaves
$\mathscr{F}_{i_{0},\ldots,i_{k}}$ defined by
\[
  \mathscr{F}_{i_{0},\ldots,i_{k}}(V)=\mathscr{F}(V\cap
  U_{i_{0},\ldots,i_{k}}),
\]
where $U_{i_{0},\ldots, i_{k}}=U_{i_{0}}\cap\cdots\cap U_{i_{k}}$.
Then, I can define an exact sequence of these sheaves following the sequence
\eqref{diag:sheaf_ex}:
\[\begin{tikzcd}
    0\rar & \mathscr{F}\rar
    & \displaystyle\prod_{i_{0}}\mathscr{F}_{i_{0}}\rar{d^{0}}
    & \displaystyle\prod_{i_{0},i_{1}}\mathscr{F}_{i_{0},i_{1}}
  \end{tikzcd}.\]
Now, how should this sequence be continued? Supposedly we would want to
find a map
\[
  d^{1}: \prod_{i_{0},i_{1}}\mathscr{F}_{i_{0},i_{1}}
  \to\prod_{i_{0},i_{1},i_{2}}\mathscr{F}_{i_{0},i_{1},i_{2}}
\]
such that $\ker d^{1}=\im d^{0}$. Let us first try to understand the sheaf
$\im d^{0}$. Suppose $\phi$ is a section in
$\displaystyle\prod\mathscr{F}_{i_{0}}(V)$. Then, the image $\psi$ under
$d^{0}$ consists of components, where the component at the index $i_{0},i_{1}$
is the difference of the component of $\phi$ at $i_{0}$ and the component of
$\phi$ at $i_{1}$ restricted to $V\cap U_{i_{0},i_{1}}$:
\[
  \psi_{i_{0},i_{1}}=\phi_{1}\vert_{V\cap U_{i_{0},i_{1}}}
  -\phi_{0}\vert_{V\cap U_{i_{0},i_{1}}},
\]
where $\phi_{0}\in\mathscr{F}(V\cap U_{i_{0}})$ and
$\phi_{1}\in\mathscr{F}(V\cap U_{i_{1}})$. Fixing one more index $i_{2}$, we
can take an alternating sum of sections restricted to
$V\cap U_{i_{0},i_{1},i_{2}}$ with these indeces to get zero. For
conciseness, I omit the restriction symbols in the following computation,
because every section is restricted to a common set.
\begin{align*}
  \psi_{i_{1},i_{2}}-\psi_{i_{0},i_{2}}+\psi_{i_{0},i_{1}}
  &=(\phi_{2}-\phi_{1})-(\phi_{2}-\phi_{0})+(\phi_{1}-\phi_{0}) \\
  &=(\phi_{2}-\phi_{2})+(\phi_{1}-\phi_{1})+(\phi_{0}-\phi_{0})=0.
\end{align*}
Thus, $\psi\in\ker(d_{V}^{1})$. Conversely, suppose $\psi\in\ker(d_{V}^{1})$
implying
\[\psi_{i_{1},i_{2}}\vert_{V\cap U_{i_{0},i_{1},i_{2}}}
-\psi_{i_{0},i_{2}}\vert_{V\cap U_{i_{0},i_{1},i_{2}}}
+\psi_{i_{0},i_{1}}\vert_{V\cap U_{i_{0},i_{1},i_{2}}}=0.\]
Take two arbitrary sections $\phi_{0}\in\mathscr{F}(V\cap U_{i_{0}})$
and $\phi_{1}\in\mathscr{F}(V\cap U_{i_{1}})$ with
$\phi_{1}\vert_{V\cap U_{i_{0},i_{1}}}-\phi_{0}\vert_{V\cap U_{i_{0},i_{1}}}
=\psi_{i_{0}, i_{1}}$. One can find two more sections (TODO)
$\phi_{2},\phi_{2}^{\prime}\in\mathscr{F}(V\cap U_{i_{2}})$ with
$\phi_{2}\vert_{V\cap U_{i_{0},i_{2}}}-\phi_{0}\vert_{V\cap U_{i_{0},i_{2}}}
=\psi_{i_{0}, i_{2}}$ and $\phi_{2}^{\prime}\vert_{V\cap U_{i_{1}, i_{2}}}
-\phi_{1}\vert_{V\cap U_{i_{1},i_{2}}}=\psi_{i_{1},i_{2}}$. Then,
\begin{align*}
  0 &= \psi_{i_{1},i_{2}}-\psi_{i_{0},i_{2}}+\psi_{i_{0},i_{1}} \\
    &= (\phi_{2}^{\prime}-\phi_{1})-(\phi_{2}-\phi_{0})+(\phi_{1}-\phi_{0}) \\
    &= \phi_{2}^{\prime}-\phi_{2}.
\end{align*}
Therefore, we have $\psi\in\im(d_{V}^{0})$. One can continue this sequence
in the same fashion by taking alternating sums of sections, thus obtaining
a resolution
\[\begin{tikzcd}
    0\rar & \mathscr{F}\rar
    & \displaystyle\prod_{i_{0}}\mathscr{F}_{i_{0}}\rar
    & \displaystyle\prod_{i_{0},i_{1}}\mathscr{F}_{i_{0},i_{1}}\rar
    & \displaystyle\prod_{i_{0},i_{1},i_{2}}\mathscr{F}_{i_{0},i_{1},i_{2}}\rar
    & \cdots
  \end{tikzcd},\]
\begin{rem}
  The above computations might seem familiar if you have seen homology
  or cohomology before. Infact, \v Cech cohomology can be seen as a singular
  cohomology of the \textbf{nerve} of the open covering $(U_{i})$ of the space
  \cite{hatcher}. The nerve of a covering is a simplicial complex, where the
  sets $U_{i}$ are the points and a non-empty intersection of $k$ of the sets
  is a $k$-simplex.
\end{rem}
The last thing I need to show is that the resolution is acyclic.
\begin{lemm}
  The sheaves
  $\displaystyle\prod_{i_{0},\ldots,i_{k}}\mathscr{F}_{i_{0},\ldots,i_{k}}$ are
  acyclic.
\end{lemm}
\begin{proof}
  By Lemma~\ref{lemm:acyclic_prod}, I only need to check that the sheaf
  $\mathscr{F}_{i_{0},\ldots,i_{k}}$ is acyclic for an arbitrary index
  $i_{0},\ldots,i_{k}$. Thus, fix an index and denote $V=U_{i_{0},\ldots,i_{k}}$
  for short. Then, consider the quasi-coherent sheaf $\mathscr{F}\vert_{V}$.
  The set $V$ is an affine open set by Lemma~\ref{lemm:affine_intersection}
  so the sheaf $\mathscr{F}\vert_{V}$ is acyclic by
  Lemma~\ref{lemm:affine_cohom_vanishes} above. Now, suppose
  $\mathscr{F}\vert_{V}$ has an injective resolution
  \[\begin{tikzcd}
      0\rar & \mathscr{F}\vert_{V}\rar & \mathscr{I}^{0}\rar
      & \mathscr{I}^{1}\rar & \mathscr{I}^{2}\rar & \cdots
    \end{tikzcd}.\]
  This is now a sequence of sheaves on $V$, and we would like to
  ``push them forward'' from $V$ to $X$ and get an injective resolution of
  $\mathscr{F}_{i_{0},\ldots,i_{k}}$. Thus, I will extend the sheaves
  $\mathscr{I}^{i}$ to $X$ by defining sheaves $\mathscr{I}_{+}^{i}$ as follows
  \[
    \mathscr{I}_{+}^{i}(U)=\mathscr{I}^{i}(V\cap U).
  \]
  Next I show that the sheaves $\mathscr{I}_{+}^{i}$ have the properties
  one would expect.
  \begin{claim}
    The sheaves $\mathscr{I}_{+}^{i}$ are quasi-coherent and injective.
  \end{claim}
  {\renewcommand{\qedsymbol}{$\blacksquare$}
    \begin{proof}
      First I show that the $\mathscr{I}_{+}^{i}$ are quasi-coherent.
      Thus, suppose $U\subseteq X$ is an affine open set. I want to show
      that $\mathscr{I}_{+}^{i}\vert_{U}$ is the sheaf associated to some
      $\mathscr{O}_{X}(U)$-module $N$. Note that since the sheaf
      $\mathscr{I}^{i}$ is a quasi-coherent sheaf, there is a
      $\mathscr{O}_{X}(V\cap U)$-module $M$ such that
      $\mathscr{I}^{i}\vert_{V\cap U}\cong\cm$. Since $\mathscr{I}_{+}^{i}(U)
      =\mathscr{I}^{i}(V\cap U)$ by definition, $N$ should consist of the same
      elements as $M$, and only difference between them is the ring that
      acts on them. In other words, we expect to be able to define $N$ as
      the restriction of scalars of $M$. This is indeed possible, since
      the inclusion $V\cap U\hookrightarrow U$ induces a ring homomorphism
      $\mathscr{O}_{X}(U)\to \mathscr{O}_{X}(V\cap U)$. We can the
      define $N$ as the restriction of scalars of $M$ along this
      homomorphism, and then $\mathscr{I}_{+}^{i}\vert_{U}=\widetilde{N}$.

      Let us now check injectivity. Suppose there is an injection
      $\mathscr{A}\hookrightarrow\mathscr{B}$ of sheaves on $X$ and a
      morphism $g:\mathscr{A}\to\mathscr{I}_{+}^{i}$. We get corresponding
      morphisms when restricting to $V$, and we can lift the restriction of
      $g$ because of the injectivity of $\mathscr{I}^{i}$.
      \[\begin{tikzcd}[row sep=large]
          \mathscr{I}^{i} & \\ \mathscr{A}\vert_{V}\uar{g\vert_{V}}\rar[hook]
          & \mathscr{B}\vert_{V}\arrow[ul, dashed, "u"']
        \end{tikzcd}\]
      Now it is easy to construct a lift $u^{+}$ of $g$ by defining it on the
      stalks. Thus, fix a point $P\in X$. If $P\in V$, then the we should
      take $u^{+}_{P}=u_{P}$. If on the other hand $P\not\in V$, then
      $u^{+}_{P}=0$, since $\left(\mathscr{I}_{+}^{i}\right)_{P}=0$. Then,
      the following triangle commutes.
      \[\begin{tikzcd}[row sep=large, column sep=large]
          \mathscr{I}_{+}^{i} & \\ \mathscr{A}\uar{g}\rar[hook]
          & \mathscr{B}\arrow[ul, "u^{+}"']
        \end{tikzcd}\]
    \end{proof}
  }
  Therefore, the sequence
  \[\begin{tikzcd}
      0\rar & \mathscr{F}_{i_{0},\ldots,i_{k}}\rar & \mathscr{I}_{+}^{0}\rar
      & \mathscr{I}_{+}^{1}\rar & \mathscr{I}_{+}^{2}\rar & \cdots
    \end{tikzcd}\]
  is an injective resolution of $\mathscr{F}_{i_{0},\ldots,i_{k}}$,
  and hence the cohomology of $\mathscr{F}_{i_{0},\ldots,i_{k}}$ can
  be computed by computing the cohomology of the sequence
  \[\begin{tikzcd}
    0\rar & \Gamma\left(\mathscr{I}_{+}^{0}\right)\rar
    & \Gamma\left(\mathscr{I}_{+}^{1}\right)\rar
    & \Gamma\left(\mathscr{I}_{+}^{2}\right)\rar & \cdots,
    \end{tikzcd}\]
  By the construction of the sheaves $\mathscr{I}_{+}^{i}$, this sequence is
  equal to
  \[\begin{tikzcd}
      0\rar & \Gamma\left(\mathscr{I}^{0}\right)\rar
      & \Gamma\left(\mathscr{I}^{1}\right)\rar
      & \Gamma\left(\mathscr{I}^{2}\right)\rar & \cdots
    \end{tikzcd},\]
  so the statement follows since $\mathscr{F}\vert_{V}$ is acyclic.
\end{proof}

\subsection{\v Cech cohomology}
Thus, we finally arrive at the definition of \v Cech cohomology.
\begin{defin}\label{def:cech}
  The \v Cech cohomology of a quasi-coherent sheaf $\mathscr{F}$ is the
  cohomology of the cochain complex
  \[\begin{tikzcd}
      \displaystyle\prod_{i_{0}}\mathscr{F}(U_{i_{0}})\rar{d^{0}}
      & \displaystyle\prod_{i_{0},i_{1}}\mathscr{F}(U_{i_{0},i_{1}})\rar{d^{1}}
      & \displaystyle\prod_{i_{0},i_{1},i_{2}}\mathscr{F}(U_{i_{0},i_{1},i_{2}})\rar{d^{2}}
      & \cdots
    \end{tikzcd},\]
  where I denote
  \[U_{i_{0},\ldots, i_{k}}=U_{i_{0}}\cap\cdots\cap U_{i_{k}}\]
  and the differentials are defined as the products of the maps
  \[d_{i_{0},\ldots,i_{k+1}}^{k}:\displaystyle\prod_{j_{0},\ldots,j_{k}}
  \mathscr{F}(U_{j_{0},\ldots,j_{k}})\to \mathscr{F}(U_{i_{0},\ldots,i_{k+1}})\]
  given by
  \[
    d_{i_{0},\ldots,i_{k+1}}^{k}(\alpha)
    =\sum_{j=0}^{k+1}(-1)^{j}\alpha_{i_{0},\ldots,\hat{i}_{j},\ldots,i_{k+1}}
    \vert_{U_{i_{0},\ldots,i_{k+1}}}.
  \]
  More specifically, the $i$th cohomology group $H^{i}(X,\mathscr{F})$
  is defined as the $k$-vector space $\ker(d^{i})/\im(d^{i-1})$ if we
  take $d^{-1}$ to be the zero-morphism.
\end{defin}
I will now use this new tool to compute some cohomology groups.
\begin{prop}\label{prop:const_sheaf}
  If $X$ is an irreducible variety and $A$ is some module,
  then for the constant sheaf $\underline{A}$,
  \begin{enumerate}[(a)]
    \item $H^{0}(X,\underline{A}) = A$,
    \item $H^{1}(X,\underline{A})=0$.
  \end{enumerate}
\end{prop}

\begin{prop}\label{prop:sky_cohom}
  If $X$ is an irreducible variety and $A$ is some module,
  then the cohomology groups $H^{i}(X,A_{P})$ of the skyscraper sheaf
  $A_{P}$ are zero for $i>0$.
\end{prop}
\begin{proof}
  TODO.
\end{proof}
