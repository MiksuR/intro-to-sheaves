\section{Sheaf cohomology}
Studying sheaves using homological algebra turns out to be surprisingly
useful in many situations. For example, knowing that there is a SES
\[
  \begin{tikzcd}
    0 \rar & \mathscr{F} \rar & \mathscr{G} \rar &
    \mathscr{H} \rar & 0
  \end{tikzcd}
\]
lets us relate the three sheaves together. By Thm.~\ref{thm:ses_equivalence}
this information is inherently \emph{local} since this sequence is exact
if and only if the corresponding sequences on stalks are exact.
Then the question is: Can we get \emph{global} information from
such exact sequences? We would hope that just as the sequence is exact
on stalks, it would also be exact on global sections:
\[
\begin{tikzcd}
  0 \rar & \Gamma(\mathscr{F}) \rar & \Gamma(\mathscr{G})
  \rar & \Gamma(\mathscr{H}) \rar & 0.
\end{tikzcd}
\]
Unfortunately, this is not the case. One can prove that this sequence
is exact at $\Gamma(\mathscr{F})$ and $\Gamma(\mathscr{G})$, but it
is not always exact at $\Gamma(\mathscr{H})$.
\begin{cat}
  We say that the global sections functor $\Gamma: \text{Sh}(X)\to \textbf{Ab}$ is left-exact but not right-exact.
\end{cat}
In more concrete terms, if we know that a morphism $f: \mathscr{F}
\to\mathscr{G}$ is injective on stalks, it is also injective on
global sections. But if the morphism is surjective on stalks, we don't
know whether or not it is surjective on global sections.
% TODO: Add an example

Although exact sequences are not completely preserved under taking
global sections, we won't give up! There might still be \emph{a way of
measuring how much exactness fails}. We could measure the
\emph{obstruction} to exactness by continuing the sequence to the right
so that the following sequence is exact.
\[
\begin{tikzcd}
  0 \rar & \Gamma(\mathscr{F}) \rar & \Gamma(\mathscr{G})
  \rar & \Gamma(\mathscr{H})
  \arrow[out=0, in=180, looseness=1.5, overlay]{dll} & \\
    & H^{1}(\mathscr{F}) \rar & H^{1}(\mathscr{G})
  \rar & H^{1}(\mathscr{H}) \rar & \cdots \\
  \cdots \rar& H^{i}(\mathscr{F}) \rar & H^{i}(\mathscr{G})
  \rar & H^{i}(\mathscr{H}) \rar & \cdots.
\end{tikzcd}
\]
This problem of extending incomplete short exact sequences appears
elsewhere in homological algebra and is generally solved by constructing
so-called \emph{derived functors}. These modules $H^{i}(-)$ given by derived
functors are then called the sheaf cohomology groups. In practise, they are
difficult to compute, so we want to find an alternative definition,
which is easier to work with. This is achieved by \emph{\v Cech cohomology}.
Next, I will explain derived functors and show how we arrived at \v Cech
cohomology from there. However, this will involve a lot of new, high-level
concepts and category theory. Understanding the philosophy behind
\v Cech cohomolgy is not essential, and one can safely skip straight to the
definition of the \v Cech cohomology groups given in Def.~\ref{def:cech}.

\subsection{Motivating \v Cech cohomology}


\subsection{Results in cohomology}

\begin{prop}\label{prop:const_sheaf}
  If $X$ is an irreducible variety, and $A$ is an abelian group,
  then for the constant sheaf $\underline{A}$,
  \begin{enumerate}[(a)]
    \item $H^{0}(X,\underline{A}) = A$,
    \item $H^{1}(X,\underline{A})=0$.
  \end{enumerate}
\end{prop}
