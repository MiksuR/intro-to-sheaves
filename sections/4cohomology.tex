\section{Sheaf cohomology}
Studying sheaves using homological algebra turns out to be surprisingly
useful in many situations. For example, knowing that there is a SES
\[
  \begin{tikzcd}
    0 \rar & \mathscr{F} \rar & \mathscr{G} \rar &
    \mathscr{H} \rar & 0
  \end{tikzcd}
\]
lets us relate the three sheaves together. By Thm.~\ref{thm:ses_equivalence}
this information is inherently \emph{local} since this sequence is exact
if and only if the corresponding sequences on stalks are exact.
Then the question is: Can we get \emph{global} information from
such exact sequences? We would hope that just as the sequence is exact
on stalks, it would also be exact on global sections:
\[
\begin{tikzcd}
  0 \rar & \Gamma(\mathscr{F}) \rar & \Gamma(\mathscr{G})
  \rar & \Gamma(\mathscr{H}) \rar & 0.
\end{tikzcd}
\]
Unfortunately, this is not the case. For example, let
$X=\mathbb{P}^{1}_{\mathbb{C}}$ and consider the sheaf morphism
\[f:\mathscr{O}_{X}\to\mathbb{C}_{P_{0}}\oplus\mathbb{C}_{P_{1}},\]
which evaluates $f$ at some points $P_{0},P_{1}\in X$. Then, the
morphism is clearly surjective on the stalks. But it is not surjective
on global sections, since the global sections of $\mathscr{O}_{X}$ are
the constant functions. In other words, the exact sequence
\[\mathscr{O}_{X}\to\mathbb{C}_{P_{0}}\oplus\mathbb{C}_{P_{1}}\to 0\]
does not give an exact sequence on global sections.

However, we have the following.
\begin{prop}
  If the sequence
  \[
  \begin{tikzcd}
    0\rar & \mathscr{F}\rar & \mathscr{G}\rar & \mathscr{H}\rar & 0
  \end{tikzcd}
  \]
  is exact, then the sequence
  \[
  \begin{tikzcd}
    0\rar & \Gamma(\mathscr{F})\rar{\alpha} & \Gamma(\mathscr{G})\rar{\beta}
    & \Gamma(\mathscr{H})
  \end{tikzcd}
  \]
  is also exact.
\end{prop}
\begin{proof}\hfill
  \begin{description}[style=nextline]
    \item[Exactness at $\Gamma(\mathscr{F})$]
          Suppose $\phi\in\ker(\alpha)$ % TODO
  \end{description}

\end{proof}
\begin{cat}
  We say that the global sections functor $\Gamma: \text{Sh}(X)\to \textbf{Ab}$ is left-exact but not right-exact.
\end{cat}

Although exact sequences are not completely preserved under taking
global sections, we won't give up! There might still be \emph{a way of
measuring how much exactness fails}. We could measure the
\emph{obstruction} to exactness by continuing the sequence to the right
so that the following sequence is exact.
\[
\begin{tikzcd}
  0 \rar & \Gamma(\mathscr{F}) \rar & \Gamma(\mathscr{G})
  \rar\dar[phantom, ""{coordinate, name=Z}] & \Gamma(\mathscr{H})
  \arrow[rounded corners, to path={ -- ([xshift=2ex]\tikztostart.east)
    |- (Z) -| ([xshift=-2ex]\tikztotarget.west) -- (\tikztotarget)},
  overlay]{dll} & \\
    & H^{1}(\mathscr{F}) \rar & H^{1}(\mathscr{G})
  \rar & H^{1}(\mathscr{H}) \rar & \cdots \\
  \cdots \rar& H^{i}(\mathscr{F}) \rar & H^{i}(\mathscr{G})
  \rar & H^{i}(\mathscr{H}) \rar & \cdots.
\end{tikzcd}
\]
This problem of extending incomplete short exact sequences appears
elsewhere in homological algebra and is generally solved by constructing
so-called \emph{derived functors}. These vector spaces $H^{i}(-)$ given by
derived functors are then called the sheaf cohomology groups. In practise,
they are difficult to compute, and thus I will define the \emph{\v Cech
  cohomology} which is a tool for computing sheaf cohomology.
In the next subsection I will introduce derived functors and deduce the
definition of \v Cech cohomology. The contents of the subsection will be
more technical than the rest of the paper, and it is probably a good idea
to skip straight to Definition~\ref{def:cech}, which can be taken as
\emph{the} definition of sheaf cohomology.

\subsection{From derived functors to \v Cech cohomology*}
Let us consider the general problem of extending a left-exact
functor $\mathcal{F}:\mathcal{A}\to\mathcal{B}$ between abelian categories
(which one may think of as categories of modules). Thus, for objects
$A,B,C$ of $\mathcal{A}$, we want to find groups $H^{i}(A),H^{i}(B),H^{i}(C)$
and connecting morphisms so that the sequence
\[
\begin{tikzcd}
  0 \rar & \mathcal{F}(A) \rar & \mathcal{F}(B)
  \rar\dar[phantom, ""{coordinate, name=Z}] & \mathcal{F}(C)
  \arrow[rounded corners, to path={ -- ([xshift=2ex]\tikztostart.east)
    |- (Z) -| ([xshift=-2ex]\tikztotarget.west) -- (\tikztotarget)},
  overlay]{dll} & \\
    & H^{1}(A) \rar & H^{1}(B)\rar & H^{1}(C) \rar & \cdots \\
  \cdots \rar& H^{i}(A) \rar & H^{i}(B)\rar & H^{i}(C) \rar & \cdots.
\end{tikzcd}
\]
is exact. The following lemma from homological algebra gives a hint of
what approach we should take to find such groups.
\begin{lemm}[Zig-zag lemma]
  Suppose $A^{\bullet},B^{\bullet},C^{\bullet}$ are cochain complexes in some
  abelian category. If there is a SES
  \[\begin{tikzcd}
      0\rar & A^{\bullet}\rar & B^{\bullet}\rar & C^{\bullet}\rar & 0
    \end{tikzcd},\]
  then there are maps between the cohomology groups of these complexes
  such that the sequence
  \[\begin{tikzcd}
    & H^{0}(A^{\bullet}) \rar & H^{0}(B^{\bullet})
    \rar\dar[phantom, ""{coordinate, name=Z}] & H^{0}(C^{\bullet})
    \arrow[rounded corners, to path={ -- ([xshift=2ex]\tikztostart.east)
      |- (Z) -| ([xshift=-2ex]\tikztotarget.west) -- (\tikztotarget)},
    overlay]{dll} & \\
    & H^{1}(A^{\bullet}) \rar & H^{1}(B^{\bullet})\rar
    & H^{1}(C^{\bullet}) \rar & \cdots \\
    \cdots \rar& H^{i}(A^{\bullet}) \rar & H^{i}(B^{\bullet})\rar
    & H^{i}(C^{\bullet}) \rar & \cdots.
    \end{tikzcd}\]
  is exact.
\end{lemm}
\begin{proof}
  The maps $H^{i}(A^{\bullet})\to H^{i}(B^{\bullet})$ and $H^{i}(B^{\bullet})
  \to H^{i}(C^{\bullet})$ are given by functoriality, and the connecting
  morphisms $H^{i}(C^{\bullet})\to H^{i+1}(A^{\bullet})$ are given by the
  snake lemma. It would be an easy but mundane exercise to check exactness
  of this sequence. I will instead use a spectral sequence argument (See
  below for more discussion on spectral sequences).

  Define the zeroth page of a spectral sequence to be the following
  double complex given by the SES of complexes.
  \[\begin{tikzcd}
      & \vdots & \vdots & \vdots & \\
      0\rar & A^{2}\rar\uar & B^{2}\rar\uar & C^{2}\rar\uar & 0 \\
      0\rar & A^{1}\rar\uar & B^{1}\rar\uar & C^{1}\rar\uar & 0 \\
      0\rar & A^{0}\rar\uar & B^{0}\rar\uar & C^{0}\rar\uar & 0 \\
      & 0\uar & 0\uar & 0\uar &
    \end{tikzcd}\]
  Since the rows are exact the first page is zero when we use the rightward
  orientation. Now, compute the first page using upward orientation.
  We get the following
  \[\begin{tikzcd}
      & \vdots & \vdots & \vdots & \\
      0\rar & H^{2}(A^{\bullet})\rar{\alpha_{2}}
      & H^{2}(B^{\bullet})\rar{\beta_{2}} & H^{2}(C^{\bullet})\rar & 0 \\
      0\rar & H^{1}(A^{\bullet})\rar{\alpha_{1}}
      & H^{1}(B^{\bullet})\rar{\beta_{1}} & H^{1}(C^{\bullet})\rar & 0 \\
      0\rar & H^{0}(A^{\bullet})\rar{\alpha_{0}}
      & H^{0}(B^{\bullet})\rar{\beta_{0}} & H^{0}(C^{\bullet})\rar & 0 \\
    \end{tikzcd}\]
  Finally, the second page we will have

  \begin{center}
  \begin{tikzpicture}[commutative diagrams/every diagram, x=2.2cm, y=1.7cm]
    \clip (0.8,0.5) rectangle (5.2,3.5);
    \node (A0) at (2,1) {$H^{0}(A^{\bullet})$};
    \node (B0) at (3,1) {$H^{0}(B^{\bullet})$};
    \node (C0) at (4,1) {$H^{0}(C^{\bullet})$};
    \node (A1) at (2,2) {$H^{1}(A^{\bullet})$};
    \node (B1) at (3,2) {$H^{1}(B^{\bullet})$};
    \node (C1) at (4,2) {$H^{1}(C^{\bullet})$};
    \node (A2) at (2,3) {$H^{2}(A^{\bullet})$};
    \node (B2) at (3,3) {$H^{2}(B^{\bullet})$};
    \node (C2) at (4,3) {$H^{2}(C^{\bullet})$};
    \node (lu) at (1,3) {$0$};
    \node (lm) at (1,2) {$0$};
    \node (rd) at (5,1) {$0$};
    \node (rm) at (5,2) {$0$};

    \path[commutative diagrams/.cd, every arrow, every label]
      (0,4) edge (A2)
      (0,3) edge (A1)
      (0,2) edge (A0)
      (lu) edge (B1)
      (lm) edge (B0)
      (1,4) edge (B2)
      (2,4) edge (C2)
      (A2) edge (C1)
      (A1) edge (C0)
      (B2) edge (rm)
      (B1) edge (rd)
      (C2) edge (6,2)
      (C1) edge (6,1)
      (C0) edge (6,0)
      (A0) edge (4,0)
      (B0) edge (5,0);
  \end{tikzpicture}
  \end{center}
\end{proof}

\subsection{Results in cohomology}

\begin{prop}\label{prop:const_sheaf}
  If $X$ is an irreducible variety, and $A$ is an abelian group,
  then for the constant sheaf $\underline{A}$,
  \begin{enumerate}[(a)]
    \item $H^{0}(X,\underline{A}) = A$,
    \item $H^{1}(X,\underline{A})=0$.
  \end{enumerate}
\end{prop}
