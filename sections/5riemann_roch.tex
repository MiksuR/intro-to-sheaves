\section{The Riemann-Roch theorem}
Equiped with sheaf cohomology, we will prove the Riemann-Roch theorem
using the methods we have learnt. But first we need to quickly review
two constructions needed to state the Riemann-Roch theorem: \emph{divisors}
and \emph{differentials}.

\subsection{Divisors and differentials}

\subsection{Proof of Riemann-Roch}
We are now able to state and prove an ``incomplete'' version of the
Riemann-Roch theorem, which we will make complete after proving Serre
duality.

\begin{lnote}
  In this section, $X$ will always denote an irreducible, non-singular,
  complete algebraic curve, and any divisor $D$ is defined on $X$.
\end{lnote}

% TODO: Give a proof that the cohomology groups are finite-dimensional.

\begin{thm}[Riemann-Roch, cohomology version]
  \label{thm:riemann_roch_cohomology}
  For every divisor $D$,
  \[
    h^{0}(X, \mathcal{L}(D))-h^{1}(X, \mathcal{L}(D))=\deg(D)+1-g,
  \]
  where $g=h^{1}(X, \mathcal{O}_{X})$.
\end{thm}
\begin{proof}
  We can use an induction argument, because any divisor $D$ can be
  obtained from the zero divisor by adding and subtracting points.
  Thus, we proceed by first proving the base case and then proving
  the induction step.

  \begin{description}[style=nextline]
    \item[base case$\big)$]
          Since $\mathcal{L}(0)=\mathscr{O}_{X}$ and $\deg(0)=0$,
          we need to verify that
          \[h^{0}(X, \mathscr{O}_{X})-h^{1}(X, \mathscr{O}_{X})=1-g.\]
          But note that the only globally defined regular functions
          on $X$ are constant by Cor.~\ref{cor:global_const},
          and thus they form a one-dimensional vector space.
          Moreover, $h^{1}(X, \mathscr{O}_{X})=g$ by definition
          so that the equality holds.
    \item[induction step$\big)$]
          In the induction step we want to relate the zeroth and first
          cohomology groups of $\mathcal{L}(D)$ to the zeroth and
          first cohomology groups of $\mathcal{L}(D+P)$, where
          $P$ is some point. To do this, we first note that
          $\mathcal{L}(D)$ is a subsheaf of $\mathcal{L}(D+P)$, since
          the orders of germs of $\mathcal{L}(D+P)$ at $P$ are allowed to
          be smaller than the orders of germs of $\mathcal{L}(D)$ at $P$.
          Thus, there is an exact sequence
          \[
          \begin{tikzcd}
            0\arrow{r} & \mathcal{L}(D)\arrow{r} & \mathcal{L}(D+P)\arrow{r}
            & Q\arrow{r} & 0,
          \end{tikzcd}
          \]
          where $Q$ is the quotient sheaf. The stalks of $Q$ are clearly
          zero away from $P$. The stalk at $P$ consists of zero and
          elements of the form $u/t^{n+1}$, where $t$ is the local
          uniformiser of $\mathscr{O}_{P}$, $u$ is a unit in
          $\mathcal{O}_{P}$, and $n$ is the order of $P$ in $D$.
          As $\mathcal{O}_{P}/(t)=k$, we can write $u=vt+r$,
          where $v\in\mathcal{O}_{P}$ and $r\in k$.
          Then,
          \[\frac{u}{t^{n+1}}=\frac{v}{t^n}+\frac{r}{t^{n+1}}.\]
          Since $v/t^n$ is an element of $\mathcal{L}(D)_{P}$, we conclude
          that every element of $\mathcal{L}(D+P)_{P}$ is equivalent to
          an element $r/t^{n+1}$ modulo $\mathcal{L}(D)_{P}$ for some
          $r\in k$. Therefore, $Q_{P}\cong k$ and $Q$ is the skyscraper sheaf!

          Now we apply our cohomology machinery on the SES
          \[
          \begin{tikzcd}
            0\arrow{r} & \mathcal{L}(D)\arrow{r} & \mathcal{L}(D+P)\arrow{r}
            & k_{P}\arrow{r} & 0
          \end{tikzcd}
          \]
          to get the following exact sequence (using
          Prop.~\ref{prop:sky_cohom}).
          \[
          \begin{tikzcd}
            0\arrow{r} & H^{0}(X, \mathcal{L}(D))\arrow{r}
            & H^{0}(X, \mathcal{L}(D+P))\arrow{r}
            & H^{0}(X, k_{P}) \\
            \arrow{r} & H^{1}(X, \mathcal{L}(D))\arrow{r}
            & H^{1}(X, \mathcal{L}(D+P))\arrow{r} & 0.
          \end{tikzcd}
          \]
          Now, by Prop.~\ref{prop:homology_dim},
          \[
          h^{0}(X,\mathcal{L}(D))-h^{0}(X, \mathcal{L}(D+P))
          +1-h^{1}(X,\mathcal{L}(D))+h^{1}(X,\mathcal{L}(D+P)) = 0.
          \]
          Therefore,
          \begin{align*}
            h^{0}(X,\mathcal{L}(D+P))&-h^{1}(X,\mathcal{L}(D+P))
            =\left(h^{0}(X,\mathcal{L}(D))-h^{1}(X,\mathcal{L}(D+P))\right)
              +1 \\
            =&\deg(D)+1-g+1\quad\text{(by induction hypothesis)} \\
            =&\deg(D+P)+1-g.
          \end{align*}
          This is exactly the induction step we wanted to prove.
          We also need to prove
          \[
            h^{0}(X,\mathcal{L}(D-P))-h^{1}(X,\mathcal{L}(D-P))
            =\deg(D-P)+1-g,
          \]
          but we can run the same argument starting with the SES
          \[
          \begin{tikzcd}
            0\arrow{r} & \mathcal{L}(D-P)\arrow{r} & \mathcal{L}(D)\arrow{r}
            & k_{P}\arrow{r} & 0.
          \end{tikzcd}
          \]
  \end{description}
\end{proof}

This form of the theorem is not the most useful one for applications,
because computing $h^{1}(X,\mathcal{L}(D))$ is not easy. Luckily, the Serre
Duality theorem --- which we will prove later --- relates zeroth and first
cohomology groups, and implies the following equality of dimensions:
\[h^{1}(X,\mathcal{L}(D))=h^{0}(X,\mathcal{L}(K_{X}-D))\]

Now, this equality lets us write the complete form of the Riemann-Roch
theorem.
\begin{thm}[Riemann-Roch]\label{thm:riemann_roch}
  For every divisor $D$,
  \[
    h^{0}(X, \mathcal{L}(D))-h^{0}(X, \mathcal{L}(K_{X}-D))=\deg(D)+1-g,
  \]
  where $g=h^{1}(X, \mathcal{O}_{X})$.
\end{thm}

\subsection{Applications}
Before proving the Serre duality theorem, I want to take some time to
look at applications of the Riemann-Roch theorem.
