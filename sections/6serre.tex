\section{Serre duality}
The rest of this paper is devoted to proving the Serre duality:
\begin{thm}[Serre Duality]
  If $X$ is an algebraic curve as in the previous section
  and $D$ is a divisor on $X$, there is an isomorphism
  \[
    H^{1}(X, \mathcal{L}(D))^{\vee}\cong H^{0}(X, K_{X}
    \otimes \mathcal{L}(D)^{\vee}).
  \]
  of $k$-vector spaces, where $K_{X}$ is the canonical divisor of $X$.
\end{thm}
We will prove the theorem by first finding a more concrete representation of
$H^{1}(X,\mathcal{L}(D))$ and then constructing a perfect pairing between
$H^{1}(X,\mathcal{L}(D))$ and $H^{0}(X,K_{X}\otimes\mathcal{L}(D)^{\vee})$,
which will give us the isomorphism.

\subsection{Concrete representation of $H^{1}(X,\mathcal{L}(D))$}
To prove the Serre duality, we do not want to directly work with the \v Cech
cohomology definition of $H^{1}(X,\mathcal{L}(D))$.
Instead we want to find some SES involving $\mathcal{L}(D)$,
take the cohomology sequence of the SES, and then use it to simplify
the definition of $H^{1}(X,\mathcal{L}(D))$. Since $\mathcal{L}(D)$ is
a subsheaf of the constant sheaf $\underline{k(X)}$, we can simply consider
the following SES.
\[
  \begin{tikzcd}
    0\arrow{r} & \mathcal{L}(D)\arrow{r} & \underline{k(X)}\arrow{r}
    & \underline{k(X)}/\mathcal{L}(D)\arrow{r} & 0,
  \end{tikzcd}
\]
which yields the following exact sequence
\[
  \begin{tikzcd}
    H^{0}(X,\underline{k(X)})\arrow{r} & H^{0}(X,\underline{k(X)}
    /\mathcal{L}(D))\arrow{r} & H^{1}(X,\mathcal{L}(D))\arrow{r}
    & H^{1}(X,\underline{k(X)}).
  \end{tikzcd}
\]
But Prop.~\ref{prop:const_sheaf} implies that
$H^{0}(X,\underline{k(X)})=k(X)$ and $H^{1}(X,\underline{k(X)})=0$
so that the exact sequence simplifies to
\[
  \begin{tikzcd}
    k(X)\arrow{r} & H^{0}(X,\underline{k(X)}/\mathcal{L}(D))\arrow{r}
    & H^{1}(X,\mathcal{L}(D))\arrow{r} & 0.
  \end{tikzcd}
\]
Let us first try to understand the space
$H^{0}(X,\underline{k(X)}/\mathcal{L}(D))$. An element of the space
is of the form $([f_{P}])_{P\in X}$, where $[f_{P}]$ is the equivalence
class of some $f_{P}\in k(X)$ modulo $\mathcal{L}(D)_{P}$.
Note that the $f_{P}$ must be related together so that the sections
satisfy sheaf axioms, and it would be easier if we didn't need to worry about
this condition. We can in fact construct an isomorphic vector space, which is
similar, but where we don't need to worry about the sheaf axioms. To see
this, first make the following observation.
\begin{prop}
  If $(f_{P})_{P\in X}\in \Gamma(\underline{k(X)})$,
  $f_{P}\in\mathscr{O}_{X,P}$ for almost all $P\in X$.
\end{prop}
\begin{proof}
  Fix an arbitrary $P\in X$. Since $\underline{k(X)}$ is obtained by
  sheafification of the constant presheaf, there must be an open
  set $U\subseteq X$ and $g\in k(X)$ such that
  $\forall Q\in U,\ [f_{Q}]=[g]$. Next we note that $g$
  is defined for almost all points of $U$ so that $g_{Q}
  \in\mathscr{O}_{X,Q}$ for almost all $Q\in V$.

  Thus, there is an open neighbourhood for every point
  of $X$ where the statement holds. Next we extend this to the whole of $X$.
  Let us construct a sequence of open sets inductively:
  \begin{enumerate}
    \item Choose an arbitrary point $P_{0}\in X$
    \item Let $U_{0}$ be the neighbourhood of $P_{0}$ constructed as above
    \item Assume we have constructed the set $U_{n}$
    \item Choose an arbitrary point $P_{n+1}$ of $X\setminus U_{n}$
    \item Let $U_{n+1}$ be the union of $U_{n}$ with the
          neighbourhood of $P_{n+1}$ constructed as above
  \end{enumerate}
  % Insert picture with X, P_0, U_0, and P_1
  Since $X$ is Noetherian, there must be $N\in\mathbb{N}$
  such that $\forall k\geq N, U_{k+1}=U_{k}$. Moreover, these sets must
  cover $X$, because otherwise the chain wouldn't end at $U_{N}$.
  Therefore, $X$ can be covered by finitely many sets where the statement
  holds.
\end{proof}
\begin{rem}
  This proposition implies that
  \[
    H^{0}(X,\underline{k(X)}/\mathcal{L}(D))\cong \bigoplus_{P\in X}
    k(X)/\mathcal{L}(D)_{P},
  \]
  which might be a helpful way of thinking this cohomology group.
\end{rem}
Now, we define a vector space of families $\{r_{P}\}_{P\in X}$, where we don't
impose any other requirement on $r_{P}$ other than that $r_{P}\in k(X)$ and
$r_{P}\in\mathscr{O}_{X,P}$ for almost all $P\in X$ (Serre calls such a
family a \emph{r\'epartition}).
% TODO: It might be wise to point out that by r_P\in O_{X,P} we
% actually mean (r_P)_P\in O_{X,P}.
Now, I claim that there is a subspace
$S\leq R$ such that $H^{0}(X,\underline{k(X)}/\mathcal{L}(D))\cong R/S$.
This isomorphism will be given by the trivial map
\[
  \varphi: H^{0}(X,\underline{k(X)}/\mathcal{L}(D))\to R/S
  :([f_{P}])_{P\in X}\mapsto [\{f_{P}\}_{P\in X}].
\]
For this map to be well-defined, $([f_{P}])_{P\in X}$ needs to be
mapped to $[0]$ whenever $f_{P}\in\mathcal{L}(D)_{P}$. Since I want
$\varphi$ also to be injective, such elements $(f_{P})_{P\in X}$ should be
the \emph{only} elements that get mapped to $[0]$. Thus, I define $S$
to be the subspace such that $\set{\{r_{P}\}_{P\in X}\mid \ord_{P}(r_{P})
  \geq -D(P)}=:R(D)$, and I claim that this is the right choice of $S$.

\begin{lemm}
  For a divisor $D$ on a curve $X$, we have the following isomorphism:
  \[
    H^{0}(X,\underline{k(X)}/\mathcal{L}(D))\cong R/R(D).
  \]
\end{lemm}
\begin{proof}
  The map $\varphi$ is a well-defined injection by construction,
  so we only need to show it is surjective. Thus,
  suppose $[\{r_{P}\}_{P\in X}]\in R/R(D)$. I want to show that equivalence
  classes $[r_{P}]$ of the components form a global section of the sheaf
  $\underline{k(X)}/\mathcal{L}(D)$. First, let $P_{1},\ldots,P_{r}$
  be the points where $D$ is non-zero and $Q_{1},\ldots,Q_{s}$ be the
  points where $r_{Q_{i}}\not\in \mathscr{O}_{X,Q_{i}}$. Then, let $P\in X$
  be an arbitrary point. We want to find an open neighbourhood $U\ni P$
  and a section $g\in k(X)$ such that $\forall Q\in U,\ [r_{Q}]=[g]$.
  There are two cases:
  \begin{description}[style=nextline]
    \item[$P\not\in\set{P_{1},\ldots,P_{r},Q_{1},\ldots,Q_{s}}\big)$]
          It follows from the definition of the points $P_{i}$ and $Q_{j}$
          that $[r_{P}]=[0]$. And if we let $U$ be the complement of
          the set $\set{P_{1},\ldots,P_{r},Q_{1},\ldots,Q_{s}}$, we see
          that the same hold for every $r_{Q}$ on $U$ so that we can simply
          choose $0\in k(X)$ as the section on the open neighbourhood $U$.
    \item[$P\in\set{P_{1},\ldots,P_{r},Q_{1},\ldots,Q_{s}}\big)$]
          First, denote $Y=\set{P_{1},\ldots,P_{r},Q_{1},\ldots,Q_{s}}
          \setminus \set{P}$ and $g=r_{P}\in k(X)$. Next, let
          $S_{1},\ldots,S_{t}$ be the points where
          $g_{S_{i}}\not\in\mathscr{O}_{X,S_{i}}$. Then, let $U$ be
          the complement of $Y\cup \set{S_{1},\ldots,S_{t}}$. As above,
          $[r_{Q}]=[0]$ for all $Q\in U$ except for $Q=P$. But since
          the points $S_{i}$ are also included in the complement, we have
          that $[g]=[0]$ away from $P$. Thus, $[r_{Q}]=[g]$ on $U$.
  \end{description}
\end{proof}
Now we can return to the SES derived above and replace
$H^{0}(X,\underline{k(X)}/\mathcal{L}(D))$ by $R/R(D)$:
\[
  \begin{tikzcd}
    k(X)\arrow{r} & R/R(D)\arrow{r} & H^{1}(X,\mathcal{L}(D))\arrow{r} & 0.
  \end{tikzcd}
\]
This exact sequence finally gives us the representation of the first
cohomology group: The second map of the sequence is a surjection
onto $H^{1}(X,\mathcal{L}(D))$. The space $k(X)$ can be seen as a subspace
of $R$ and it is thus the kernel of the map. Such a surjection maps $R/R(D)$
onto the space $R\,/\left(R(D)+k(X)\right)$, and thus we get the isomorphism
\[H^{1}(X,\mathcal{L}(D))\cong R/\left(R(D)+k(X)\right).\]
Now, the dual space $H^{1}(X,\mathcal{L}(D))^{\vee}$ is simply the space
of linear functionals on $R$, which vanish on $R(D)$ and $k(X)$!

\subsection{Constructing a pairing}
Next I will construct a bilinear form
\[
  \langle -,-\rangle:\diffs\times H^{1}(X,\mathcal{L}(D))\to k.
\]
Note that the space $\diffs$ consists of differential forms $\omega$
such that $(\omega)\geq D$.
%But first we want to understand the space $H^{0}(X,K_{X}
%\otimes\mathcal{L}(D)^{\vee})$ better.
%Consider the stalk of the sheaf
%$K_{X}\otimes\mathcal{L}(D)^{\vee}$ at a point $P$:
%\[
%  \left(K_{X}\otimes\mathcal{L}(D)^{\vee}\right)_{P}
%  =(K_{X})_{P}\otimes_{\mathscr{O}_{X}}\mathcal{L}(D)^{\vee}_{P}
%  =D_{k}(\mathscr{O}_{X})\otimes_{\mathscr{O}_{X}}\mathcal{L}(D)^{\vee}_{P}
%  =D_{k}\left(\mathcal{L}(D)^{\vee}_{P}\right).
%\]
Now, define the bilinear form as follows.
\[
  \langle\omega,r\rangle=\sum_{P\in X}\res_{P}(r_{P}\omega),
\]
where $r=[\{r_{P}\}_{P\in X}]\in R\,/\left(R(D)+k(X)\right)$. The sum is
well-defined, because the term $\res_{P}(r_{P}\omega)$ can be non-zero only
when $P$ is a point such that $D(P)\neq 0$ or $r_{P}\not\in\mathscr{O}_{X,P}$.
Otherwise, $r_{P},f\in\mathscr{O}_{X,P}$, if we write $\omega=f\,dt$
where $t$ is a local uniformiser at $P$. This clearly implies that the
coefficients of the negative terms in the serier expansion of $r_{P}f$
are all zero.

Now, Serre duality will follow from showing that the map
\[
  i_{D}:\diffs\to H^{1}(X,\mathcal{L}(D))^{\vee}
  :\omega\mapsto\langle\omega,-\rangle
\]
is a bijection. But first we need to confirm that $i_{D}$ actually maps
differentials to elements of $H^{1}(X,\mathcal{L}(D))^{\vee}$. For a
differential $\omega\in\diffs$, $i_{D}(\omega)$ is of course a linear
functional on $R$, but we need to check that it vanishes on $R(D)$ and
$k(X)$. If $r\in R(D)$, then $(r_{P}\omega)=(r_{P})+(\omega)\geq -D+D=0$
and thus $\res_{P}(r_{P}\omega)=0$ by the same argument as in the last
paragraph. And if $r\in k(X)$, then $\langle r,\omega\rangle=0$ by
Thm.~\ref{thm:residue}. Thus, $i_{D}(\omega)\in H^{1}(X,\mathcal{L}(D))^{\vee}$
for every $\omega\in\diffs$.

Now, we can proceed to prove the bijectivity of $i_{D}$, starting with
injectivity.
\begin{prop}\label{prop:injectivity}
  The map $i_{D}:\diffs\to H^{1}(X,\mathcal{L}(D))^{\vee}$ is an injection.
\end{prop}
\begin{proof}
  The map is injective if its kernel is trivial. Thus, suppose
  $i_{D}(\omega)=0$. I will show that for every $P\in X$,
  $\ord_{P}(\omega)=\infty$, which implies $\omega=0$.
  Assume to the contrary, so that there is a point $P\in X$ where
  $\ord_{P}(\omega)$ is bounded. But now we can construct a r\'epartition
  $r=[\{r_{Q}\}_{Q\in X}]$ such that $r_{Q}=0$ when $Q\neq P$ and
  $r_{P}=1/t^{\ord_{P}(\omega)+1}$, where $t$ is a local uniformiser at $P$.
  Then,
  \[
    i_{D}(\omega)(r)=\sum_{Q\in X}\res_{Q}(r_{Q}\omega)=\res_{P}(r_{P}\omega)
  \]
  Since $\ord_{P}(r_{P}\omega)=\ord_{P}(r_{P})+\ord_{P}(\omega)
  =-\ord_{P}(\omega)-1+\ord_{P}(\omega)=-1$, we have that $i_{D}(\omega)(r)$ is
  non-zero, which contradicts the assumption that $i_{D}(\omega)=0$.
\end{proof}

To prove that $i_{D}$ is surjective, I will do a small trick by taking
the union
\[
  \bigcup_{D\text{ divisor on }X}H^{1}(X,\mathcal{L}(D))^{\vee}=:\duals,
\]
defining a $k(X)$-vector space structure on the union $\duals$,
and proving that the map
\[
  i_{D}^{\ast}:D_{k}(k(X))\to\duals:\omega\mapsto \langle\omega, -\rangle
\]
is a surjective map of $k(X)$-vector spaces. Then, the surjectivity
of $i_{D}$ will follow.

For $f\in k(X)$ and $\alpha\in\duals$, I define the scalar product
$f\alpha$ as the linear functional given by $r\mapsto\alpha(fr)$,
i.e. it is the linear functional that evaluates the linear
functional $\alpha$ at $fr$. It needs to be checked that $f\alpha\in\duals$.
Since $f\alpha$ is clearly a linear functional on $R$, I need to just check
that it vanishes on $k(X)$ and $R(D)$ for some divisor $D$. Thus, suppose
$\alpha\in H^{1}(X,\mathcal{L}(D_{1}))^{\vee}$ and $f\in k(X)$. Then, $f\alpha$
vanishes on $k(X)$, because $\alpha$ does. Moreover, if we suppose
that $f\in H^{0}(X,\mathcal{L}(D_{2}))$ and $r\in R(D_{1}-D_{2})$, then
\[
  \ord_{P}(fr_{P})=\ord_{P}(f)+\ord_{P}(r_{P})\geq -D_{2}(P)-(D_{1}-D_{2})(P)
  =-D_{1}(P),
\]
which implies that $fr\in R(D_{1})$. Now, since $\alpha$ vanishes on
$R(D_{1})$, I have shown that $f\alpha$ vanishes on $R(D_{1}-D_{2})$.
Now that we see $f\alpha\in\duals$, it is easy to verify that this
scalar product defines a vector space structure on $\duals$.

The map
\[
  i_{D}^{\ast}:D_{k}(k(X))\to\duals:\omega\mapsto \langle\omega, -\rangle
\]
introduced above is a linear map of $k(X)$-vector spaces. Additivity is
easy to see, and homogeneity follows from the way we defined scalar product
in $\duals$: For $r\in H^{1}(X,\mathcal{L}(D))$,
\[
  i_{D}^{\ast}(f\omega)(r)=\langle f\omega, r\rangle = \langle\omega, fr\rangle
  =\left(f\langle\omega,-\rangle\right)(r)=f\,i_{D}^{\ast}(\omega)(r).
\]
\begin{lemm}
  The linear map $i_{D}^{\ast}:D_{k}(k(X))\to\duals$ is a surjection.
\end{lemm}
\begin{proof}
  We know that $D_{k}(k(X))$ is 1-dimensional vector space. I will show
  that $\dim(\duals)\leq1$, which implies that the map must be surjective,
  as it is clearly not trivial. Assume for a contradiction that there are two
  linearly independent elements $\alpha, \alpha^{\prime}\in\duals$. One can
  see that there is a divisor $D$ such that $\alpha, \alpha^{\prime}
  \in H^{1}(X,\mathcal{L}(D))^{\vee}$. If we denote by $\Delta_{n}$ some
  divisor of degree $n$, where $n\geq 0$, then $f,g
  \in H^{0}(X,\mathcal{L}(\Delta_{n}))$ implies that $f\alpha,g\alpha^{\prime}
  \in H^{1}(X,\mathcal{L}(D-\Delta_{n}))^{\vee}$ by the argument above.
  Thus, we can define the map
  \[
    \psi: H^{0}(X,\mathcal{L}(\Delta_{n}))\oplus
    H^{0}(X,\mathcal{L}(\Delta_{n}))\to
    H^{1}(X,\mathcal{L}(D-\Delta_{n}))^{\vee}:f+g\mapsto
    f\alpha+g\alpha^{\prime}.
  \]
  Since $\alpha$ and $\alpha^{\prime}$ are linearly independent, this
  map is an injection: $\psi(f+g)=0\implies f\alpha+g\alpha^{\prime}=0
  \implies f,g=0$. This implies an inequality in dimension:
  \[h^{1}(X,\mathcal{L}(D-\Delta_{n}))\geq 2h^{0}(X,\mathcal{L}(\Delta_{n})).\]
  Next I will apply the cohomology version of Riemann-Roch theorem
  (Thm.~\ref{thm:riemann_roch_cohomology}) on both sides of the inequality.

  First, the Riemann-Roch theorem gives us
  $h^{1}(X,\mathcal{L}(D-\Delta_{n}))=-\deg(D-\Delta_{n})+g-1
  +h^{0}(X,\mathcal{L}(D-\Delta_{n}))$. If we choose $n>\deg(D)$, then
  $\deg(D-\Delta_{n})<0$ so that $h^{0}(X,\mathcal{L}(D-\Delta_{n}))=0$
  by Prop.~\ref{prop:neg_deg_divisor_sections}. Thus,
  \[
    h^{1}(X,\mathcal{L}(D-\Delta_{n}))=-\deg(D)+\deg(\Delta_{n})+g-1=n-\deg(D)
    +g-1.
  \]
  Next, Riemann-Roch gives $h^{0}(X,\mathcal{L}(\Delta_{n}))\geq
  \deg(\Delta_{n})+1-g=n+1-g$.
  These two imply that
  \[
    n-\deg(D)+g-1\geq 2n+2-2g\iff 3g-3-\deg(D)\geq n.
  \]
  Since $n$ was arbitrary, we can choose it to be larger than $3g-3-\deg(D)$,
  and hence we get a contradiction.
\end{proof}

Now, the surjectivity of $i_{D}$ follows easily from this lemma.
\begin{prop}\label{prop:surjectivity}
  The map $i_{D}:\diffs\to H^{1}(X,\mathcal{L}(D))^{\vee}$ is an surjection.
\end{prop}
\begin{proof}
  Let $\alpha\in H^{1}(X,\mathcal{L}(D))^{\vee}$. By the above lemma,
  there is $\omega\in D_{k}(k(X))$ such that $i_{D}^{\ast}(\omega)=\alpha$.
  It must be shown that $\omega\in\diffs$. Assume to the contrary so that
  there is a point $P$ such that $\ord_{P}(\omega)<D(P)$. As in the proof
  of Prop.~\ref{prop:injectivity}, we can consider the r\'epartition
  $r=\{r_{P}\}_{P\in X}$ such that $r_{Q}=0$ for $Q\neq P$ and $r_{P}
  =1/t^{\ord_{P}(\omega)+1}$, where $t$ is a local uniformiser at $P$.
  Then, $\ord_{P}(r_{P}\omega)=-1$ and $\langle\omega,r\rangle\neq 0$.
  This is a contradiction, because $r\in R(D)$, but $\langle\omega,-\rangle$
  should vanish on $R(D)$.
\end{proof}

Therefore, we finally arrive at the Serre Duality theorem!
\begin{thm}[Serre Duality]
  If $X$ is an algebraic curve as in the previous section
  and $D$ is a divisor on $X$, there is an isomorphism
  \[
    H^{1}(X, \mathcal{L}(D))^{\vee}\cong H^{0}(X, K_{X}
    \otimes \mathcal{L}(D)^{\vee}).
  \]
  of $k$-vector spaces, where $K_{X}$ is the canonical divisor of $X$.
\end{thm}
\begin{proof}
  Combining Propositions \ref{prop:injectivity} and \ref{prop:surjectivity}
  shows that the linear map
  \[i_{D}:\diffs\to H^{1}(X,\mathcal{L}(D))^{\vee}\]
  is a bijection so that it defines an isomorphism between the two spaces.
\end{proof}
